% nottsclassic --%
%           
% Copyright (c) 2016 Lukas C. Bossert | William Leveritt
%  
% This work may be distributed and/or modified under the
% conditions of the LaTeX Project Public License, either version 1.3
% of this license or (at your option) any later version.
% The latest version of this license is in
%   http://www.latex-project.org/lppl.txt
% and version 1.3 or later is part of all distributions of LaTeX
% version 2005/12/01 or later.
%
%!TEX program = xelatex
\documentclass[a4paper,
10pt,
english
]{ltxdoc}
\input{nottsclassic-preamble.tex}
\NewEnviron{bibbsp}[1]
  {\par\medskip\noindent\footnotesize
  \begin{tikzpicture}
    \node[inner sep=0pt] (box) {\parbox[t]{\textwidth}{%
      \begin{minipage}{.07\textwidth}
      \hspace{.5em}\scriptsize \textbf{\Cref{#1}}\hfill
      \end{minipage}%
      \begin{minipage}{.9\textwidth}
      \BODY
      \end{minipage}\hfill}%
    };
    \draw[codeblue,line width=2pt] 
      ( $ (box.north east) + (-5pt,3pt) $ ) -- ( $ (box.north east) + (0,3pt) $ ) -- ( $ (box.south east) + (0,-3pt) $ ) -- + (-5pt,0);
    \draw[codeblue,line width=2pt] 
      ( $ (box.north west) + (5pt,3pt) $ ) -- ( $ (box.north west) + (0,3pt) $ ) -- ( $ (box.south west) + (0,-3pt) $ ) -- + (5pt,0);
  \end{tikzpicture}\par\medskip%
}

\begin{document}
\title{\texttt{nottsclassic} -- \\\texttt{bib\LaTeX}-style of the Classics department \footnote{The development of the code is done at \url{https://github.com/LukasCBossert/biblatex-nottsclassic}.}}
\author{Lukas C. Bossert\thanks{\href{mailto:lukas@digitales-altertum.de}{lukas@digitales-altertum.de}} \and William Leveritt}
\date{Version: 0.1 (2016-06-09)}
 \maketitle
\begin{abstract}

 \end{abstract}


\begin{multicols}{2}
\footnotesize\parskip=0mm \tableofcontents
\end{multicols}


\section{Usage}
 \DescribeMacro{nottsclassic}  The name of the bib\LaTeX-style is  |nottsclassic| has to be activated in the preamble. 

\begin{lstlisting}
\usepackage[style=nottsclassic,%
					*@\meta{further options}@*]{biblatex}
\bibliography*@\marg{|bib|-file}@*
\end{lstlisting}


At the end of your document you can write the command |\printbibliography| to print 
the bibliography. 
Since |nottsclassic| supports different citations of various texts like from ancient authors and from modern scholars we suggest to have them listed in separated bibliographies. 
Further information are found below   (\cref{bibliographie}).

\section{Overview}\label{overview}
Following there is a quick overview of possible options of the style |nottsclassic|. 
Contrary to the alphabetically ordered description later (\cref{options-description}) they here are listed by topic.
Furthermore you can -- at your own risk -- also use the conventional |bib|\LaTeX-options which are related of indent, etc. 
For that please see the excellent documentation of  |bib|\LaTeX.

\subsection{Preamble options}\label{preamble_options}

\subsubsection{Manner of citing}

\DescribeMacro{noabbrv}

 
\subsubsection{Global bibliography settings}


\subsection{Entry Options}


\subsection{Cite commands}\label{cite-commands}
\DescribeMacro{\cite}%
As always citing is done with \cs{cite}:
\begin{lstlisting}
\cite*@\oarg{prenote}\oarg{postnote}\marg{bibtex-key}%@*
\end{lstlisting}

\meta{prenote} sets a short preliminary note (e.\,g. \enquote{Vgl.}) and \meta{postnote} is usually used for page numbers.
If only one optional argument is used then it is \oarg{postnote}.
\begin{lstlisting}
\cite*@\oarg{postnote}\marg{bibtex-key}%@*
\end{lstlisting}
The \meta{bibtex-key} corresponds to the key from the bibliography file.

\DescribeMacro{\cites}
If one wants to cite several authors or works a very convenient way is the following using the \cs{cites}-command:
\begin{lstlisting}
\cites(pre-prenote)(post-postnote)*@\oarg{prenote}\oarg{postnote}\marg{bibtex-key}@*%
 																	*@\oarg{prenote}\oarg{postnote}\marg{bibtex-key}@*%
 																	*@\oarg{prenote}\oarg{postnote}\marg{bibtex-key}\ldots@*
\end{lstlisting}
 
\DescribeMacro{\parencite}
Sometimes a citation has to be put in parentheses. 
Therefore we implemented the command \cs{parencite}:
\begin{lstlisting}
\parencite*@\oarg{postnote}\marg{bibtex-key}%@*
\end{lstlisting} 
This cite command takes care of the correct corresponding parentheses and brackets.
Especially in |@Inreference| citations the parentheses are changing to (square) brackets.
The example shown in \cref{faq:inreference} makes it clear.

\DescribeMacro{\parencites}
Of course there is also the possibility to cite several authors/works in parentheses.
This is done with \cs{parencites}:
\begin{lstlisting}
\parencites(pre-prenote)(post-postnote)*@\oarg{prenote}\oarg{postnote}\marg{bibtex-key}@*%
 																			*@\oarg{prenote}\oarg{postnote}\marg{bibtex-key}@*%
 																			*@\oarg{prenote}\oarg{postnote}\marg{bibtex-key}\ldots@*
\end{lstlisting}
 
\DescribeMacro{\textcite}
Beside the listed \cs{cite} commands above there is a third way of citing:
\cs{textcite} is useful if the author should be mentioned in the text and
the remaining components like year and page will immediately follow in parentheses. 
\begin{lstlisting}
\textcite*@\oarg{postnote}\marg{bibtex-key}%@*
\end{lstlisting} 

\DescribeMacro{\textcites}
And again there is also a \cs{textcites} in case of several authors: 
  \begin{lstlisting}
\textcites(pre-prenote)(post-postnote)*@\oarg{prenote}\oarg{postnote}\marg{bibtex-key}@*%
 																			*@\oarg{prenote}\oarg{postnote}\marg{bibtex-key}@*%
 																			*@\oarg{prenote}\oarg{postnote}\marg{bibtex-key}\ldots@*
\end{lstlisting}

\DescribeMacro{\citeauthor}\DescribeMacro{\citetitle}\label{citeauthor}%
Furthermore and additionally to the ›normal‹ \cs{cite}-commands one can also cite only the author or the work title in the text and in the footnotes.
\begin{lstlisting}
\citeauthor*@\oarg{prenote}\oarg{postnote}\marg{bibtex-key}%@*
\end{lstlisting} 
  and for the works 
\begin{lstlisting}
\citetitle*@\oarg{prenote}\oarg{postnote}\marg{bibtex-key}%@*
\end{lstlisting} 
For further information cf. \cref{fullnames}.


 \section{Bibliography}\label{bibliographie}
 \DescribeMacro{\printbibliography}
As long as you don’t use the option\DescribeMacro{seenote} |seenote|---for 
which a final bibliography is not needed---you will need to print you cited entries in a bibliography 
at a certain place in your document.
It can be useful to differentiate your bibliography and divide it e.\,g. into a bibliography 
with ancient authors and one with modern scholars.
Additionally you can have a bibliography with the |shorthand| shortcuts or all abbreviated journal titles, etc.

How the different bibliographies can be set up is explained now:
Let’s assume you want to have a bibliography with the ancient authors and one with modern scholars.
Since the entries of the ancient authors have the field |keyword={ancient}| (or should have it) this is done quite easy.

But first we define the heading of the whole  bibliography:
\begin{lstlisting}
\printbibheading[%
							heading=bibliography,%
							%heading=bibnumbered,% if you want it numbered
							title={Bibliography}] %heading for bibliography
\end{lstlisting}
You can give any title you would like to give (|title = |\marg{any title}).

The next step is to set up the bibliography for the ancient authors.

\begin{lstlisting}
\printbibliography[%
							keyword=ancient,%
							heading=subbibliography,
							%heading=subbibnumbered,% if you want it numbered
							title={Ancient authors and works}]
\end{lstlisting}
We tell the bibliography just to contain the entries which have have |ancient| in the field |keywords| (line 2).


Finally the bibliography for modern scholars:
\begin{lstlisting}
\printbibliography[%
							notkeyword=ancient,%
							notkeyword=corpus,%
							heading=subbibliography,
							%heading=subbibnumbered,% if you want it numbered
							title={Secondary literature}]
\end{lstlisting}
This time we exclude all entries which have |ancient| or |corpus| in the field |keywords|. 
That’s it.
(Don't be surprised about the line |notkeyword=corpus| which excludes entries with special |shorthand| labels, a further bibliography part with all the |shorthands| is described below.).


\begin{refsection}
\nocite{*}
Now have a look how it looks like with all the entries we explained above.
\begin{bsp}
\renewcommand\bibfont{\normalfont\footnotesize}
\printbibheading[%
							heading=bibliography,%
							title={Bibliography}] %heading for bibliography

\printbibliography[%
							keyword=ancient,%
							heading=subbibliography,
							title={Ancient authors and works}]

\printbibliography[%
							notkeyword=ancient,%
							notkeyword=corpus,%
							heading=subbibliography,
							title={Secondary literature}]
\end{bsp}

You can create as many bibliographies as you wish each with an other keyword if you like.
Or you can make a bibliography with all the |shorthands| used in your text---for that we use |keyword= {corpus}| (line 2):
\begin{lstlisting}
\printbibliography[%
						keyword=corpus,%
						heading=subbibliography,
					 	title={Abbreviation and corpora}]
\end{lstlisting}
Now the bibliography only lists the used entries which have |corpus| in the field |keywords|:
\begin{bsp}
\printbibliography[%
						keyword={corpus},
          				heading=subbibliography,
            			title={Abbreviation of corpora}]\label{bib:corpus}
\end{bsp}

Note: If you want to separate in your bibliography author-year labels from |shorthand| labels
 you should insure yourself that bibliography entries which contain a |shorthand| denomination 
are set with a keyword either |ancient|, |corpus| or something else, to guarantee that there is 
no bibliographical shortcut wrongly sorted in the bibliography.


Furthermore you can have a bibliography for all the abbreviated journal titles and series to have the abbreviation and its long form.
For journals it works like this:
\begin{lstlisting}
\printbiblist[%
					heading=subbibliography,
					title={Abbreviation of journals}]{shortjournal}
\end{lstlisting}

\begin{bsp}
\printbiblist[%
					heading=subbibliography,
					title={Abbreviation of journals}]{shortjournal}
\end{bsp}

For series it is done like this:
\begin{lstlisting}
\printbiblist[%
					heading=subbibliography,
					title={Abbreviation of series}]{shortseries}
\end{lstlisting}


\begin{bsp}
\printbiblist[heading=subbibliography,title={Abbreviation of series}]{shortseries}\end{bsp}
\end{refsection}



\clearpage
\begin{multicols}{2}\footnotesize
\lstlistoflistings
\end{multicols}

\end{document}
