% archaeologie --%
%            biblatex for archaeologists, 
%				historians and philologists
% Copyright (c) 2016 Lukas C. Bossert | William Leveritt
%  
% This work may be distributed and/or modified under the
% conditions of the LaTeX Project Public License, either version 1.3
% of this license or (at your option) any later version.
% The latest version of this license is in
%   http://www.latex-project.org/lppl.txt
% and version 1.3 or later is part of all distributions of LaTeX
% version 2005/12/01 or later.
%!TEX program = xelatex
\documentclass[a4paper,
10pt,
english
]{ltxdoc}
\usepackage{libertine}
\renewcommand*\ttdefault{lmvtt}

\usepackage[					% use  for bibliography
	backend=biber,
	style=archaeologie,
	lstpublishers,
	lstlocations,
	lstabbrv,
]{biblatex}
\renewcommand\bibfont{\normalfont
\footnotesize}
\usepackage{metalogo}
\usepackage{hologo}
\usepackage{babel}
\usepackage{coolthms}
\usepackage[					% advanced quotes
%	strict=true,					% 	- warning are errors now
autostyle=true,%
]{csquotes}
\usepackage{multicol}
\setlength{\columnsep}{1.5cm}
\setlength{\columnseprule}{0.2pt}

\usepackage[ 
%	headsepline, 
%	footsepline,
%	plainfootsepline, 
%markcase=upper, 
automark, 
draft=false,
]{scrlayer-scrpage} 
\pagestyle{scrheadings}
\clearscrheadfoot
	\ihead{\normalfont\footnotesize \texttt{bib}\LaTeX-style \texttt{archaeologie 2.0} (c) by Lukas C. Bossert | Johannes Friedl}%
\rofoot{\large\footnotesize  \textbf{\sffamily \thepage}}

\usepackage[%
			%flushmargin, %
%			marginal,
			ragged,%
			hang, %
			bottom%
			]{footmisc} %Fussnoten


\usepackage{framed}
\usepackage{enumitem}
\setlength{\parindent}{0pt}
\setlength{\parskip}{6pt plus 2pt minus 2pt}
\setenumerate[1]{label=(\alph*),leftmargin=*,nolistsep,parsep=\parskip}
\usepackage{changepage}
\makeindex

\defbibheading{empty}{}


\listfiles
%\EnableCrossrefs
%\CodelineIndex
%\RecordChanges
\addbibresource{archaeologie.bib}
%\addbibresource{archaeologie-ancient.bib}
%\addbibresource{antike-corpora.bib} %Version 2.0
\usepackage{caption}

\usepackage{listings}
\usepackage{chngcntr}
\AtBeginDocument{\counterwithin{lstlisting}{section}}

\usepackage{xcolor}
\definecolor{codeblue}{RGB}{0,65,137}
\definecolor{codegreen}{RGB}{147,193,26}
\definecolor{codegray}{rgb}{0.5,0.5,0.5}
\definecolor{codepurple}{rgb}{0.58,0,0.82}
\definecolor{backcolour}{rgb}{0.95,0.95,0.92}

\newenvironment{bsp}{\begin{framed}\begin{footnotesize}
\begin{adjustwidth}{.3cm}{.3cm}}{\end{adjustwidth}
\end{footnotesize}\end{framed}}
%------
\usepackage{environ}
\usepackage[tikz]{bclogo}
\usepackage{tikz}
\usetikzlibrary{calc}
%\NewEnviron{bibbsp}[1]
%  {\par\medskip\noindent\footnotesize
%  \begin{tikzpicture}
%    \node[inner sep=0pt] (box) {\parbox[t]{\textwidth}{%
%      \begin{minipage}{.15\textwidth}
%      \hspace{.5em}\scriptsize \textbf{\cref{#1}}\hfill
%      \end{minipage}%
%      \begin{minipage}{.83\textwidth}
%      \BODY
%      \end{minipage}\hfill}%
%    };
%    \draw[codeblue,line width=2pt] 
%      ( $ (box.north east) + (-5pt,3pt) $ ) -- ( $ (box.north east) + (0,3pt) $ ) -- ( $ (box.south east) + (0,-3pt) $ ) -- + (-5pt,0);
%    \draw[codeblue,line width=2pt] 
%      ( $ (box.north west) + (5pt,3pt) $ ) -- ( $ (box.north west) + (0,3pt) $ ) -- ( $ (box.south west) + (0,-3pt) $ ) -- + (5pt,0);
%  \end{tikzpicture}\par\medskip%
%}
%-----

\newcommand{\printbib}[2][4em]{%
\begingroup
\begin{bibbsp}{#2}
\footnotesize
\begin{refsection}
\setlength{\labwidthsameline}{#1} 
\nocite{#2}
\printbibliography[heading=none]
\end{refsection}
\end{bibbsp}
\endgroup
}

\newcommand{\printbiball}[2][4em]{%
\begingroup
\setlength{\labwidthsameline}{#1} 
\begin{bibbsp}{#2}
\footnotesize%
\begin{itemize}%[\bfseries]
\begin{refsection}%
\nocite{#2}%
\item[English:]{\printbibliography[heading=none]}
\item[German:]\foreignlanguage{ngerman}{\printbibliography[heading=none]}
\item[Italian:]\foreignlanguage{italian}{\printbibliography[heading=none]}
\item[French:]\foreignlanguage{french}{\printbibliography[heading=none]}
\item[Spanish:]\foreignlanguage{spanish}{\printbibliography[heading=none]}
\end{refsection}
\end{itemize}%
\end{bibbsp}
\endgroup
}

 \DeclareCaptionFormat{listing}{#1#2#3}
 \captionsetup[lstlisting]{format=listing,
 singlelinecheck=false, margin=0pt, font={sf},size=footnotesize}
\lstdefinestyle{bibentry}{%
	language=[LaTeX]TeX,
    backgroundcolor=\color{backcolour},   
    commentstyle=\color{codegreen},
    keywordstyle=\color{codeblue},
    numberstyle=\tiny\color{codegray},
    stringstyle=\color{codepurple},
    escapeinside={*@}{@*},          % if you want to add LaTeX within your code
    texcsstyle=*\color{codeblue},
    morekeywords={cites, parencites, parencite, textcite, textcites, citeauthor, citetitle,@String,
    						@Article, @Book,@Collection,@Proceedings,@Reference,@Thesis,
    						@Inproceedings,@Talk,@Review,@Inreference,@Incollection,
    						},
    basicstyle=\ttfamily\footnotesize,
    breakatwhitespace=false,         
    breaklines=true,   
    numberbychapter=false,   
    captionpos=b,                    
    keepspaces=true,               
	%framexleftmargin=5mm, 
	frame=shadowbox,
	rulesepcolor=\color{codeblue},
    numbers=left,                    
    numbersep=5pt,            
    showspaces=false,                
    showstringspaces=false,
    showtabs=false,                  
    tabsize=2,
    literate=
            *{\{}{{{\color{codegreen}{\{}}}}{1}
            {\}}{{{\color{codegreen}{\}}}}}{1}
            {[}{{{\color{codegreen}{[}}}}{1}
            {]}{{{\color{codegreen}{]}}}}{1},
}
\renewcommand{\lstlistingname}{Example}%rename caption
\renewcommand{\lstlistlistingname}{List of examples}%rename caption

 \lstdefinestyle{code}{%
	language=[LaTeX]TeX,
    backgroundcolor=\color{white},   
    commentstyle=\color{codegreen},
    keywordstyle=\color{codeblue},
    numberstyle=\small\color{codegray},
    stringstyle=\color{codepurple},
	escapeinside={*@}{@*},
    texcsstyle=*\color{codeblue},
    morekeywords={},
    basicstyle=\ttfamily\footnotesize,
    breakatwhitespace=false,         
    breaklines=true,                 
    captionpos=b,                    
    keepspaces=true,                 
    %numbers=left,                    
    stepnumber=1,
    numbersep=5pt,            
    showspaces=false,                
    showstringspaces=false,
    showtabs=false,                  
    tabsize=2,
      morekeywords={cites, parencites, parencite, textcite, textcites, citeauthor, citetitle,
    						@String,
    						@Article, @Book,@Collection,@Proceedings,@Reference,@Thesis,
    						@Inproceedings,@Talk,@Review,@Inreference,@Incollection,
    						},
    literate=
            *{\{}{{{\color{codeblue}{\{}}}}{1}
            {\}}{{{\color{codeblue}{\}}}}}{1}
            {[}{{{\color{codeblue}{[}}}}{1}
            {]}{{{\color{codeblue}{]}}}}{1},
}
%%% Always I forget this so I created some aliases
\def\ContinueLineNumber{\lstset{firstnumber=last}}
\def\StartLineAt#1{\lstset{firstnumber=#1}}
\let\numberLineAt\StartLineAt

\lstset{style=code}
\lstMakeShortInline[style=code]{|}


\usepackage{hyperxmp}
%\usepackage{hyperref}
\hypersetup{					% setup the hyperref-package options
	pdftitle={bib\LaTeX-archaeologie},	% 	- title (PDF meta)
	pdfsubject={This citation-style covers the citation and bibliography rules of the German Archaeological Institute. 
Various options are available to change and adjust the outcome according to one's own preferences. 
The style is compatible with the English, German, Italian, Spanish and French languages, since all bibstrings used are defined in each language.},% 	- subject (PDF meta)
	pdfauthor={Lukas C. Bossert, Johannes Friedl},	% 	- author (PDF meta)
	pdfauthortitle={},
	pdfcopyright={This work may be distributed and/or modified under the
conditions of the LaTeX Project Public License, either version 1.3
of this license or (at your option) any later version.},
	pdfhighlight=/N,
	pdfdisplaydoctitle=true,
	pdfdate={\the\year-\the\month-\the\day}
	pdflang={de},
	pdfcaptionwriter={Lukas C. Bossert},
	pdfkeywords={biblatex, archaeology, humanities},
%	pdfproducer={XeLaTeX},
	pdflicenseurl={http://www.latex-project.org/lppl.txt},
	plainpages=false,			% 	- 
  colorlinks   = true, %Colours links instead of ugly boxes
  urlcolor     =  codeblue, %Colour for external hyperlinks
  linkcolor    = codeblue, %Colour of internal links
  citecolor   = black, %Colour of citations
  	linktoc=page,
  	pdfborder={0 0 0},			% 	-
	breaklinks=true,			% 	- allow line break inside links
	bookmarksnumbered=true,		%
	bookmarksopenlevel=4,
	bookmarksopen=true,		%
	final=true	% = true, nur bei web-Dokument!! (wichtig!!)
}
 \crefformat{lstlisting}{#2example\ #1#3}
 \Crefformat{lstlisting}{#2 #1#3}
\crefmultiformat{lstlisting}{#2examples #1#3}{; #2#1#3}{; #2#1#3}{; #2#1#3}
\Crefmultiformat{lstlisting}{#2 #1#3}{; #2#1#3}{; #2#1#3}{; #2#1#3}
\Crefrangeformat{lstlisting}{#3examples #1#4--#5examples #2#6}
\crefrangeformat{lstlisting}{#3examples #1#4--#5examples #2#6}

\NewEnviron{bibbsp}[1]
  {\par\medskip\noindent\footnotesize
  \begin{tikzpicture}
    \node[inner sep=0pt] (box) {\parbox[t]{\textwidth}{%
      \begin{minipage}{.07\textwidth}
      \hspace{.5em}\scriptsize \textbf{\Cref{#1}}\hfill
      \end{minipage}%
      \begin{minipage}{.9\textwidth}
      \BODY
      \end{minipage}\hfill}%
    };
    \draw[codeblue,line width=2pt] 
      ( $ (box.north east) + (-5pt,3pt) $ ) -- ( $ (box.north east) + (0,3pt) $ ) -- ( $ (box.south east) + (0,-3pt) $ ) -- + (-5pt,0);
    \draw[codeblue,line width=2pt] 
      ( $ (box.north west) + (5pt,3pt) $ ) -- ( $ (box.north west) + (0,3pt) $ ) -- ( $ (box.south west) + (0,-3pt) $ ) -- + (5pt,0);
  \end{tikzpicture}\par\medskip%
}

\begin{document}
\title{\texttt{nottsclassic} -- \\\texttt{bib\LaTeX}-style of the Classics department \footnote{The development of the code is done at \url{https://github.com/LukasCBossert/biblatex-nottsclassic}: 
Comments and critics are welcome.
}}
\author{Lukas C. Bossert\thanks{\href{mailto:lukas@digitales-altertum.de}{lukas@digitales-altertum.de}} \and William Leveritt}
\date{Version: 0.1 (2016-06-09)}
 \maketitle
\begin{abstract}

 \end{abstract}


\begin{multicols}{2}
\footnotesize\parskip=0mm \tableofcontents
\end{multicols}


\section{Usage}
 \DescribeMacro{nottsclassic}  The name of the bib\LaTeX-style is  |nottsclassic| has to be activated in the preamble. 

\begin{lstlisting}
\usepackage[style=nottsclassic,%
					*@\meta{further options}@*]{biblatex}
\bibliography*@\marg{|bib|-file}@*
\end{lstlisting}


At the end of your document you can write the command |\printbibliography| to print 
the bibliography. 
Since |nottsclassic| supports different citations of various texts like from ancient authors and from modern scholars we suggest to have them listed in separated bibliographies. 
Further information are found below   (\cref{bibliographie}).

\section{Overview}\label{overview}
Following there is a quick overview of possible options of the style |nottsclassic|. 
Contrary to the alphabetically ordered description later (\cref{options-description}) they here are listed by topic.
Furthermore you can -- at your own risk -- also use the conventional |bib|\LaTeX-options which are related of indent, etc. 
For that please see the excellent documentation of  |bib|\LaTeX.

\subsection{Preamble options}\label{preamble_options}
\subsubsection{Notation of names}
\DescribeMacro{bibfullname}
In the bibliography full names of authors and/or editors are shown cf. \cref{bibfullname}.

\DescribeMacro{fullnames}
Writes only first and family names of authors or editors when they are cited with \cs{citeauthor}\marg{bibtex-key} cf. \cref{fullnames}.

\DescribeMacro{initials}
First names are abbreviated keeping digraphs and trigraphs cf. \cref{initials}.

\DescribeMacro{lastnames}
Writes only family names of authors or editors when they are cited with \cs{citeauthor}\marg{bibtex-key}.
In footnotes this is default cf.  \cref{lastnames}.

\DescribeMacro{scshape}
Cited names are shown with small capital letters cf. \cref{scshape}.
Bibliography entries with |option={ancient}| or |option={frgancient}| (\cref{ancient,frgancient}) are not touched by this option.

\subsubsection{Manner of citing}

\DescribeMacro{noabbrv}
By default the short titles of journals and series (|shortjournal| and |shortseries|) are shown in the bibliography.
With this option full titles are printed instead (|journaltitle| and |series|) cf. \cref{noabbrevs}.

\DescribeMacro{publisher}
All locations and the publisher is shown. 
It also changes the format of the edition and the first print cf. \cref{publisher}.


 
\subsubsection{Global bibliography settings}

\DescribeMacro{width}
|width={value}| defines the bibliography width between label and reference cf. \cref{width}.



\subsection{Entry Options}
A single bibliography entry can contain a value in its |options|-field.
Depending on the option it changes the behaviour how that entry is cited cf. \cref{options-bibentry,beispiele}. 
Beside their distinct properties all of these options have in common that the separating comma between citation and page record is missing. 
Actually this concerns citation of ancient texts and corpora where usually the |shorthand|-field is printed in citations.

\DescribeMacro{ancient}
The entry is an ancient source (e.\,g. Cicero, Plutarch, etc) cf. \cref{ancient}.

\DescribeMacro{frgancient}
The entry is a fragmented ancient source (e.\,g. Festus) cf. \cref{frgancient}.

\DescribeMacro{corpus} 
Only the |shorthand|-field is printed.
This is needed especially for corpora of inscriptions or coins (CIL, AE, RIC, etc.) cf. \cref{corpus}.

\changes{v1.1}{2015/06/04}{New options added in summary.}

\subsection{Cite commands}\label{cite-commands}
\DescribeMacro{\cite}%
As always citing is done with \cs{cite}:
\begin{lstlisting}
\cite*@\oarg{prenote}\oarg{postnote}\marg{bibtex-key}%@*
\end{lstlisting}

\meta{prenote} sets a short preliminary note (e.\,g. \enquote{Vgl.}) and \meta{postnote} is usually used for page numbers.
If only one optional argument is used then it is \oarg{postnote}.
\begin{lstlisting}
\cite*@\oarg{postnote}\marg{bibtex-key}%@*
\end{lstlisting}
The \meta{bibtex-key} corresponds to the key from the bibliography file.

\DescribeMacro{\cites}
If one wants to cite several authors or works a very convenient way is the following using the \cs{cites}-command:
\begin{lstlisting}
\cites(pre-prenote)(post-postnote)*@\oarg{prenote}\oarg{postnote}\marg{bibtex-key}@*%
 																	*@\oarg{prenote}\oarg{postnote}\marg{bibtex-key}@*%
 																	*@\oarg{prenote}\oarg{postnote}\marg{bibtex-key}\ldots@*
\end{lstlisting}
 
\DescribeMacro{\parencite}
Sometimes a citation has to be put in parentheses. 
Therefore we implemented the command \cs{parencite}:
\begin{lstlisting}
\parencite*@\oarg{postnote}\marg{bibtex-key}%@*
\end{lstlisting} 
This cite command takes care of the correct corresponding parentheses and brackets.
Especially in |@Inreference| citations the parentheses are changing to (square) brackets.
The example shown in \cref{faq:inreference} makes it clear.

\DescribeMacro{\parencites}
Of course there is also the possibility to cite several authors/works in parentheses.
This is done with \cs{parencites}:
\begin{lstlisting}
\parencites(pre-prenote)(post-postnote)*@\oarg{prenote}\oarg{postnote}\marg{bibtex-key}@*%
 																			*@\oarg{prenote}\oarg{postnote}\marg{bibtex-key}@*%
 																			*@\oarg{prenote}\oarg{postnote}\marg{bibtex-key}\ldots@*
\end{lstlisting}
 
\DescribeMacro{\textcite}
Beside the listed \cs{cite} commands above there is a third way of citing:
\cs{textcite} is useful if the author should be mentioned in the text and
the remaining components like year and page will immediately follow in parentheses. 
\begin{lstlisting}
\textcite*@\oarg{postnote}\marg{bibtex-key}%@*
\end{lstlisting} 

\DescribeMacro{\textcites}
And again there is also a \cs{textcites} in case of several authors: 
  \begin{lstlisting}
\textcites(pre-prenote)(post-postnote)*@\oarg{prenote}\oarg{postnote}\marg{bibtex-key}@*%
 																			*@\oarg{prenote}\oarg{postnote}\marg{bibtex-key}@*%
 																			*@\oarg{prenote}\oarg{postnote}\marg{bibtex-key}\ldots@*
\end{lstlisting}

\DescribeMacro{\citeauthor}\DescribeMacro{\citetitle}\label{citeauthor}%
Furthermore and additionally to the ›normal‹ \cs{cite}-commands one can also cite only the author or the work title in the text and in the footnotes.
\begin{lstlisting}
\citeauthor*@\oarg{prenote}\oarg{postnote}\marg{bibtex-key}%@*
\end{lstlisting} 
  and for the works 
\begin{lstlisting}
\citetitle*@\oarg{prenote}\oarg{postnote}\marg{bibtex-key}%@*
\end{lstlisting} 
For further information cf. \cref{fullnames}.


\subsection{Entries with @String}\label{string}
The citation rules of the German Archaeological Institute instruct to abbreviate journals and series according to a given list.\footnote{\url{http://www.dainst.org/documents/10180/70593/03_Liste+abzukürzender+Zeitschriften_quer.pdf/2646d351-8e5d-4e8b-8acd-54f3c272d3ff} <2016-06-06>}
For this purpose we provide a list with bibliography macros which refer to these abbreviations. 
These abbreviations can be included by loading package option |lstabbrv| (cf. \cref{abbrv}). 
Besides there are two further lists with |@String| macros cf. \cref{lstpublishers,lstlocations}.

\DescribeMacro{@String} The style |nottsclassic| respects the guidelines of the German Archaeological Institute 
and is therefore compatible with the given abbreviations of journals and series.
To minimize the susceptibility to errors and to omit unnecessary typing of sometimes very long journal titles |nottsclassic| works with so-called |@Strings|.
The advantage of these |@Strings| is that several bibliography entries can be defined by one globally given value. 
The |@String| is loaded at begin of the |bib|-file, therefore all |@Strings| have to be previous to all other bibliography entries.
 
To use this offer of simplification the bibliography fields for the title of the journal (|journaltitle|) and its short form (|shortjournal|), 
as well as the name of the series (|series|) and its shortform  (|shortseries|) have to be written with a |@String|.
In \cref{abbrv-lists} there is a list with all the abbreviations given by the German Archaeological Institute, 
in which the |@String| (with endings |-short| for |shortjournal| or |shortseries|) are listed in the left column.  
A |@String| has to be written \emph{without} any curly brackets.\footnote{If you use \emph{JabRef} in its non-coding window, 
then you have to write \#|AyasofyaMuezYil|\#. JabRef converts this internally to a |@String| and omits the \# in the coding window. 
\emph{BibDesk} provides such conversion as well by pressing |cmd-R| which enables direct BibTeX typing without enclosing curly brackets.}

An example shows how to use it:
\begin{lstlisting}[style=bibentry,label=Koyunlu1990,caption={{@}Article\{Koyunlu1990,…\} }]
@Article{Koyunlu1990,
  author       = {Koyunlu, A.},
  title        = {Die Bodenbelage und der Errichtungsort der Hagia Sophia},
  journaltitle = AyasofyaMuezYil,		%@String used
  shortjournal = AyasofyaMuezYil-short,		%@String used
  volume       = {11},
  pages        = {147--156},
  year         = {1990},
}
\end{lstlisting}

That article appeared within a rather unusual journal, 
which should be abbreviated with ›AyasofyaMüzYıl‹.

To save the time to look for the special character and insert ›ı‹ manually 
it is written in the |@String| with an ›i‹ (for further information see \cref{abbrv-lists}) but will be replaced after compiling with the correct character:

\printbib[6em]{Koyunlu1990}

Whether using the provided abbreviation list with |lstabbrv| or filling the |journaltitle| and |shortjournal| fields manually, 
|nottsclassic| uses by default short titles if defined.
The default embedding of such abbreviations can be switched off, of course.
In case you use the package option |noabbrv| in the preamble (see \cref{noabbrevs}), then the output changes as follows:
\begin{bibbsp}{Koyunlu1990}
\parbox[t]{2cm}{Koyunlu 1990} \parbox[t]{9cm}{A. Koyunlu, 
Die Bodenbelage und der Errichtungsort der Hagia Sophia, {\color{red}Ayasofia Müzesi yıllığı. Annual of Ayasofya Museum} 11, 1990, 147–156}
\end{bibbsp}
Nevertheless the advantage of our abbreviation list lies in the possibility to create a separate bibliography 
with all the abbreviations of used journal titles and series (see \cref{bibliographie}) without being prone to citation differences and typing errors.

However, if a journal or a series is \emph{not} included in the list (\cref{abbrv-lists}) 
then this journal/series will \emph{not} be abbreviated and converted to full title in curly brackets in the respective field e.\,g. |journaltitle=|\marg{title of the journal} or |series=|\marg{name of the series}. 
Therefore the field content will not be printed. 
At least Bib\TeX{} gives a warning in its log which can be checked.\footnote{For example something like \enquote{\enquote{journaltitle} in entry \meta{entrykey} cannot be null, deleting it}.}
For the following examples we use |@String| whenever it is clever and possible.

Lastly we want to point out that |@Strings| can also be used partly as following shows:

\begin{lstlisting}
@Incollection{Mundt2015,
  ...
  location     = Berlin #{ and Boston}, %@String partly used
  ...
}
\end{lstlisting}

Each time you want to leave the |@String| environment and enter the curly bracket environment (and reverse) make use of a hash \# to concatenate elements.  

\section{Details of optional preferences}\label{options-description}
Following we give a more detailed insight into the various options of |nottsclassic| 
and show their results on the bases of concrete examples.
Changes made by these options are {\color{red}coloured in red}.

\subsection{Preamble options}\label{options-preamble}
Optional preferences in the preamble are loaded within the package |bib|\LaTeX:
\begin{lstlisting}
\usepackage[%				
		backend=biber,	% activates biber which is done by default anyway 
										%	(but will give an error if not done here)
		style=nottsclassic, 	% loads the style *@|nottsclassic|@*
		inreferences=true,		% option *@|inreferences|@* is loaded
		lstabbrv							% option *@|lstabbrv|@* is loaded as well
		]{biblatex}
\end{lstlisting}
In this example the style |nottsclassic| is loaded with options |inreferences| and |lstabbrv|. 
Now, manual entries don't appear in author-year-style anymore and journal/series |@string|-macros are enabled.
By the way, it doesn't matter if you write |inreferences| or |inreferences=true|.

Each of the listed options is disabled by default even if we strongly recommend to use in particular the additional bibliographies and |@string| lists. 
All remaining options are rather a matter of taste.

Despite to the overview section (\cref{overview}) the following list is arranged in alphabetic order.

\subsubsection{bibancient}\label{bibancient}
\DescribeMacro{bibancient}
In case of citing ancient authors and their works you can do it with common \cs{cite}-commands.
Exclusively for this case we included a modification that respects the different citation of ancient authors and works.
With the option |bibancient| you load an additional bibliography called |bibliography-bibancient.bib| in which we inserted almost 600 ancient authors and works with their abbreviation according to The New Pauly/Thesaurus Linguae Latinae.
For the complete list of those see \cref{list-bibancient}.
Using these pre-sets is recommended because it will guarantee a high level of consistency and minimize error-proneness.

You can cite the authors or works with their |bibtex-key| which you find in bold in the left column of the list. 
Authors and works are separated in the |bibtex-key| by a colon.
The entry on the right marks the |shorthand| which will be printed in your paper.

Let us make it clear with an example:
\begin{refsection}
With the loaded option |bibancient| you write  

\begin{lstlisting}
\cite[3,2,5--7]{Apul:met} 
\end{lstlisting}

and will get in return after compiling: 

\cite[3,2,5--7]{Apul:met}.
\end{refsection}

The corresponding bibliography-entry looks like this
\begin{lstlisting}[style=bibentry,label=Apul:met,caption={{@}Book\{Apul:met,…\} }]
@Book{Apul:met,
  author      = {Apuleius Madaurensis, Lucius},
  title       = {metamorphoses},
  shorthand   = {Apul. met.},
  shortauthor = {Apuleius},
  keywords    = {ancient},
  options     = {ancient},
}
\end{lstlisting}
All entries in the mentioned additional bibliography contain the line |keywords = {ancient}|.
With that you can print all ancient authors in a separated bibliography by typing:
\begin{lstlisting}
\printbibliography[keyword=ancient]
\end{lstlisting}
\printbib{Apul:met}

\begin{refsection}
By means of the |bibtexkey| (e.g. |Apul:met|) you can also cite only authors or titles like this: 

\begin{lstlisting}
\citeauthor{Apul:met} remarks in \citetitle{Apul:met}...
\end{lstlisting}

will give
\begin{bsp}
\citeauthor{Apul:met} remarks in \citetitle{Apul:met}...
\end{bsp}
\end{refsection}

\changes{v1.5}{2016/05/31}{Extra Bibliographie}

\subsubsection{bibcorpora}\label{bibcorpora}
\DescribeMacro{bibcorpora}
This loads an additional bibliography which contains the most important corpora for ancient studies, so you can cite them right away in your document without creating a new bibentry by yourself. 
These corpora are listed in \cref{list-bibcorpora}. 
Advantage and usage mostly correspond to |bibancient|, 
so have a look at \cref{bibancient} for details.

\subsubsection{inreferences}\label{inreferences}

\DescribeMacro{inreferences}  
One peculiarity of citation rules of the German Archaeological Institute 
is the way manuals and encyclopaedias should be cited when not represented in author-year-style. 
|nottsclassic| allows to cite these manuals in footnotes or text as full citation in this special manner.
Single requirement is that the bibliography entry is an |@Inreference|  (cf. \cref{inreference}).
 
An example makes it clear: 
\begin{lstlisting}[style=bibentry,label=Nieddu1995,caption={{@}Inreference\{Nieddu1995,…\} }]
@Inreference{Nieddu1995,
  author    = {Nieddu, Giuseppe},
  title     = {Dei Consentes},
  booktitle = LTUR-short,
  pages     = {9--10},
  year      = {1995},
  volume    = {2},
}
\end{lstlisting}

\begin{refsection}
There are two ways to display |\cite{Nieddu1995}|:

\begin{bsp}
\begin{enumerate}
\item by default it prints author-year:
\cite{Nieddu1995}
\item with the option |inreferences| it will change to:
LTUR 2 (1995) 9-10 s. v. Dei Consentes (G. Nieddu)
\end{enumerate}
\end{bsp}

If \oarg{postnote} is defined with columns/page number, e.\,g. |\cite[9]{Nieddu1995}|, 
the position of \oarg{postnote} changes:

\begin{bsp}
\begin{enumerate} 
\item by default it prints in author-year:
\cite[9]{Nieddu1995}
\item with the option |inreferences| it appears like:
LTUR 2 (1995) 9 s. v. Dei Consentes (G. Nieddu)
\end{enumerate}
\end{bsp}

To mention an important point, activating |inreferences| causes the cited |@Inreference| entry to be automatically \emph{omitted} in the (final) bibliography,
because the entry is fully cited before in the footnotes (also stated in the German Archaeological Institute rules).

If the option is not used (|inreferences=false|) the entry will look like this in the bibliography:
\end{refsection}

\printbib[6em]{Nieddu1995} 

\subsubsection{lstabbrv}\label{abbrv}
\DescribeMacro{lstabbrv}
If you want to benefit from the above mentioned method with |@String| (cf. \cref{string}) 
you have to activate the option called |lstabbrv| (list of abbreviations) in the preamble.
Once activated the additional bibliography |nottsclassic-lstabbrv.bib| is loaded. 
In this bibliography all abbreviations listed in \cref{abbrv-lists} are stored, for further details of usage see \cref{string}.

\subsubsection{lstlocations}\label{lstlocations}
\DescribeMacro{lstlocations}
This loads an additional bibliography with |@Strings| of locations used to print out their correct exonym in the selected language. 
In that case you are not forced to change location spelling when switching the language. 
(Otherwise it is necessary to adjust location names like \emph{Rome} to \emph{Rom} or \emph{Roma} 
in your potentially multiple-used bibliography each time you change the language of your scientific text).
For details on the locations list cf. \cref{list-locations}.

\subsubsection{lstpublishers}\label{lstpublishers}
\DescribeMacro{lstpublishers}
Activates the additional bibliography file |nottsclassic-lstpublishers.bib| with |@Strings| 
of publishers which can be used to print out their correct name. 
Benefits are similar to the other lists mentioned above and in \cref{string}.
For the list cf. \cref{list-publishers}




\subsubsection{seenote}\label{seenote}
\DescribeMacro{seenote}
Even if author-year-citation seems to be commonly accepted in Ancient Studies in the meantime you may want to use a traditional citation style. 
For this purpose you can switch to the other allowed citation rule by the German Archaeological Institute
which works like this:
If you cite a work for the first time in a footnote |nottsclassic| will print a full cite which contains all bibliography elements.
Henceforward each following citation is printed as short cite and will additionally refer to the footnote where the first cite was done.
Bibliography entries with |options={ancient}| are excluded form this speciality and are cited as always.

You can use the cite-commands \cs{cite(s)} and \cs{parencite(s)} but \cs{textcite(s)} 
will behave like \cs{cite(s)} because |seenote| actually just checks for occurrences in footnotes and does not refer to cites in running text.

We give an example, first citation of 
\begin{lstlisting}
\footnote{\cite{Ball2013}}
\end{lstlisting}

displays

\begin{bsp}
L. F. Ball – J. J. Dobbins, Pompeii Forum Project. Current thinking on the Pompeii Forum, 117/3, 2013, 461–492.
\end{bsp}

Second citation of
\begin{lstlisting}
\footnote{\cite[470]{Ball2013}}
\end{lstlisting}

ends up in

\begin{bsp}
Ball – Dobbins loc. cit. (see n. [number of the footnote with full citation]) 470.
\end{bsp}
\changes{v1.5}{2016/05/31}{Rückverweis}

\subsubsection{translation}\label{translation}
\DescribeMacro{translation}
Once this option is activated the original title, the original language and the translator of the work are printed (|origtitle|, |origlanguage|, |translator|).
For ancient texts and fragments (|options={ancient}| or |options={frgancient}|) this is default, 
so they will always be printed with original title, language and translator.

An example will be clarifying.
The bibliographical entry |Lefebvre2011| contains following fields:
\begin{lstlisting}[style=bibentry,label=Lefebvre2011,caption={{@}Book\{Lefebvre2011,…\}}]
@Book{Lefebvre2011,
  author       = {Lefebvre,Henri},
  title        = {The Production of Space},
  publisher    = {Blackwell Publishing Ltd},
  location     = {Maien, MA and Oxford and Victoria},
  year         = {2011},
  edition      = {30},
  origlocation = {Oxford},
  origyear     = {1991},
  origtitle    = {La production de l’espace},
  origlanguage = {french},
  translator   = {Donald Nicholson-Smith},
}
\end{lstlisting}

The bibliography result is:
\printbib[6em]{Lefebvre2011}

By activating option |translation| it will change to:

\begin{bibbsp}{Lefebvre2011}%
\parbox[t]{2cm}{Lefebvre 2011} \parbox[t]{9cm}{H. Lefebvre,  The Production of Space, 
{\color{red} La production de l’espace, trans. from French by D. Nicholson-Smith} \textsuperscript{30}(Oxford 1991; repr. Maien, MA 2011)}
\end{bibbsp}
 
However, it works not only with entries like |@Book| but also with e.\,g. |@Article|:

\begin{lstlisting}[style=bibentry,label=Lefebvre1977,caption={{@}Article\{Lefebvre1977,…\} }]
@Article{Lefebvre1977,
  author       = {Lefebvre, Henri},
  title        = {Die Produktion des städtischen Raums},
  journaltitle = {ARCH+},
  volume       = {34},
  pages        = {52--57},
  year         = {1977},
  translator   = {Franz Hiss and Hans-Ulrich Wegener},
  origlanguage = {french},
  number       = {9},
  origtitle    = {Introduction à l'espace urbain},
}
\end{lstlisting}

Once again the bibliography entry alters:

\begin{bibbsp}{Lefebvre1977}%
\parbox[t]{2cm}{Lefebvre 1977} \parbox[t]{9cm}{H. Lefebvre, 
Die Produktion des stätischen Raums, \emph{Introduction à l’espace urbain}, 
{\color{red} trans. from French by F. Hiss -- H.-U. Wegener}, ARCH+ 34/9, 1977, 52–57}
\end{bibbsp}

\subsubsection{inreferences}\label{inreferences}
\DescribeMacro{inreferences}  
There is the possibility to cite inreferences in the footnote as a full citation.
It is only required that the bibliography-entry is an |@Inreference|  (cf. \cref{inreference}).
 
Another example makes it clear: 
\begin{lstlisting}[style=bibentry,label=Nieddu1995,caption={{@}Inreference\{Nieddu1995,…\} }]
@Inreference{Nieddu1995,
  author    = {Nieddu, Giuseppe},
  title     = {Dei Consentes},
  booktitle = LTUR-short,
  pages     = {9--10},
  year      = {1995},
  volume    = {2},
}
\end{lstlisting}
  \begin{refsection}
There are two ways of the display of |\cite{Nieddu1995}|:
 \begin{enumerate}
 \item by default it will give:  %Nieddu 1995
 \cite{Nieddu1995}
 \item with the option |inreferences| it will change to:
 LTUR 2 (1995) 9-10 s. v. Dei Consentes (G. Nieddu)
 % \cite{Nieddu_1995}
  \end{enumerate}

If the \oarg{postnote} is defined with the columns/page number, (e.\,g. |\cite[9]{Nieddu1995}|), 
then it will change the position for the \oarg{postnote}:
% \begin{bsp}
\begin{enumerate} 
 \item by default it will give:  %Nieddu 1995
 \cite[9]{Nieddu1995}
 \item with the option |inreferences| it will change to:
  LTUR 2 (1995) 9 s. v. Dei Consentes (G. Nieddu)
  \end{enumerate}
%\end{bsp}
With the activated option   (|inreferences|) the cited entries which are defined as a |@Inreference| will be automatically omitted in the (final) bibliography,
because the entries are fully cited before in the footnotes.

If the option is not used (|=false|) the entry will look like this in the bibliography:
  \end{refsection}

\printbib[6em]{Nieddu1995} 

\subsubsection{yearseries}\label{yearseries}
\DescribeMacro{yearseries}
The option |yearseries| leads to a different position of the fields |series| and |number|.
The |series| of a |@Book| or |@Collection| is now printed \emph{after} the year.
An example with an |@Incollection| demonstrates the effect of this option:
 
\begin{lstlisting}[style=bibentry,label=Mundt2015,caption={{@}Incollection\{Mundt2015,…\}}]
@Incollection{Mundt2015,
  author       = {Mundt, Felix},
  title        = {Der Mensch, das Licht und die Stadt},
  subtitle     = {Rhetorische Theorie und Praxis antiker und humanistischer Städtebeschreibung},
  pages        = {179--206},
  editor       = {Therese Fuhrer and Felix Mundt and Jan Stenger},
  booktitle    = {Cityscaping},
  booksubtitle = {Constructing and Modelling Images of the City},
  publisher    = WdG,
  location     = Berlin #{ and Boston}, %@String partly used
  year         = {2015},
  series       = Philologus-long #{ Supplement},
  number       = {3},
  shortseries  = Philologus-short #{ Suppl.},
}
\end{lstlisting}

Without any option activated it will look like this:
\printbib[5em]{Mundt2015}
 
By activating |yearseries| it will change to:
\begin{bibbsp}{Mundt2015}
\parbox[t]{1.7cm}{Mundt 2015} \parbox[t]{9cm}{F. Mundt, Der Mensch, das Licht und die Stadt. Rhetorische Theorie und Praxis antiker und humanistischer Städtebeschreibung, in: T. Fuhrer -- F. Mundt -- J. Stenger (ed.), Cityscaping. Constructing and Modelling Images of the City (Berlin 2015) {\color{red}Philologus Suppl. 3,} 179–206}
\end{bibbsp}

\subsubsection{fullnames / lastnames}\label{fullnames}\label{lastnames}
\DescribeMacro{fullnames}
\DescribeMacro{lastnames}
Every time you mention authors in the running text it is possible to cite them 
directly with their names (\cs{citeauthor}\marg{bibtex-key}) or their works  (\cs{citetitle}\marg{bibtex-key}) 
having the benefit that they will be linked to your bibliography (cf. \cref{citeauthor}).

By default the author's name is printed with abbreviated first name\footnote{Usually only the first letter, but setting the option |initials| to true it might change (cf. \cref{initials}).} and last name.
If you prefer to have full names printed (in running text, not in the bibliography!) switch on the option |fullnames|.
If you want in contrast the authors to be shorten to their last names use |lastnames|.

The following example illustrates it:

\begin{lstlisting}[style=bibentry,label=Boehmer1985,caption={{@}Article\{Boehmer1985,…\} }]
@Article{Boehmer1985,
  author       = {Boehmer, Rainer Michael and Wrede, Nadja},
  title        = {Astragalspiele in und um Warka},
  journaltitle = BaM,
  shortjournal = BaM-short,
  volume       = {16},
  pages        = {399--404},
  year         = {1985},
}
\end{lstlisting}

Let's assume you would like to write:

\begin{refsection}
\begin{lstlisting}
*@\ldots , this is also shown by @*\citeauthor{Boehmer1985} 
*@ in their latest article @*\citetitle{Boehmer1985}*@.@* \end{lstlisting}

After compiling it will look by default like this using the 
etting |fullnames| \ref{name:a} or |lastnames| \ref{name:b}:

\begin{bsp}
\begin{enumerate}
\item\label{name:a} 
\ldots , this is also shown by {\color{red}Rainer Michael Boehmer  und Nadja Wrede} in their latest article \emph{Astragalspiele in und um Warka} (1985).
\item\label{name:b}
\ldots , this is also shown by {\color{red}Boehmer und  Wrede} in their latest article \emph{Astragalspiele in und um Warka} (1985).
\end{enumerate}
\end{bsp}
\end{refsection}

\printbib[9.5em]{Boehmer1985}

Two things are left to be noticed: 
\begin{enumerate}
\item Citing an author in a footnote prints always last names, no matter which citing option is chosen. 
\item There is a slightly different behaviour if you use  \cs{citeauthor} or \cs{citetitle}  with ancient authors and work titles (|options={ancient}|).
Instead of printing the field |author| which contains usually the full ancient name the field |shortauthor| is considered in which you can record the more common name of the ancient author.
Ancient work titles will be printed without the year in parentheses. Both is demonstrated in the following example: Based on the bibliography entry
\end{enumerate}
\begin{lstlisting}[style=bibentry,label=Quint:inst,caption={{@}Book\{Quint:inst,…\} }]
@Book{Quint:inst,
  author       = {Fabius Quintilianus, Marcus},
  title        = {Ausbildung des Redners},
  subtitle     = {Institutio oratoria},
  publisher    = WBG,
  location     = {Darmstadt},
  year         = {2015},
  edition      = {6},
  origlanguage = {latin},
  translator   = {Rahn, Helmut},
  shorthand    = {Quint. inst.},
  shortauthor  = {Quintilian},
  keywords     = {ancient},
  options      = {ancient},
}
\end{lstlisting}

and the following statement

\begin{refsection}
\begin{lstlisting}
*@\ldots\ and @*\citeauthor{Quint:inst}*@ names in @*\citetitle{Quint:inst} 
*@ the necessary physical qualities of an orator, too.@*
\end{lstlisting}
 
we obtain the result:

\begin{bsp}
\ldots\ and \citeauthor{Quint:inst} names in \citetitle{Quint:inst} the  necessary physical qualities of an orator, too.
\end{bsp}
\end{refsection}

\printbib[5em]{Quint:inst}

\subsubsection{yearinparens}\label{yearinparens}
\DescribeMacro{yearinparens}%
As the options name evokes the publication year of the cited entries 
(|year| or year from |date|) will be put in parentheses,
in footnotes as well as in the bibliography. 
The ›Klammerregel‹ (correct alternation of different brackets) will be respected.

In the case of |\cite[475]{Ball2013}|, instead of
\begin{bsp} 
Ball – Dobbins 2013, 475 
\end{bsp}

now we get

\begin{bsp}
Ball – Dobbins {\color{red}(}2013{\color{red})}, 475.
\end{bsp}


\subsubsection{scshape}\label{scshape}
\DescribeMacro{scshape}
You can also change the look of your citations. 
With |scshape| author names are set to small capitals---in footnotes and in the bibliography.

Entries without |author| or |editor| setting but with a defined
 |label| (\cref{unknown}) are excluded from this option
because |label| is not an author name but a self-defined expression with varying purposes.
Further excluded are ancient authors (|options={ancient}| or |options={frgancient}|).

By default---to quote the just established example |\cite[475]{Ball2013}|---we have again

\begin{bsp} \cite[475]{Ball2013} \end{bsp}

But with |schape| it will turn into:

\begin{bsp}
{\scshape {\color{red}Ball – Dobbins}} 2013, 475
\end{bsp}

And since the entry looks like this

\begin{lstlisting}[style=bibentry,label=Ball2013,caption={{@}Article\{Ball2013,…\} }]
@Article{Ball2013,
author       = {Larry F. Ball and John J. Dobbins},
title        = {Pompeii Forum Project},
subtitle     = {Current Thinking on the Pompeii Forum},
journaltitle = AJA,
shortjournal = AJA-short,
volume       = {117},
pages        = {461--492},
year         = {2013},
doi          = {10.3764/aja.117.3.0461},
eprint       = {10.3764/aja.117.3.0461},
eprinttype   = {jstor},
month        = {July},
number       = {3},
}
\end{lstlisting}


the output in the bibliography changes from:

\printbib[8em]{Ball2013}

to 

\begin{bibbsp}{Ball2013}
\parbox[t]{3cm}{{\scshape \color{red}Ball – Dobbins} 2013}\parbox[t]{8cm}{%
L. F. Ball – J. J. Dobbins, Pompeii Forum Project. Current Thinking on the Pompeii Forum, AJA 117/3, 2013, 461–492\newline
...}
\end{bibbsp}

\subsubsection{bibfullname}\label{bibfullname}
\DescribeMacro{bibfullname}
This will show the full name of an author and/or editor in the bibliography. 
By default first names are abbreviated.

Without any options the entry

\begin{lstlisting}[style=bibentry,label=Osland2016,caption={{@}Article\{Osland2016,…\} }]
@Article{Osland2016,
  author       = {Osland, Daniel},
  title        = {Abuse or Reuse?},
  subtitle     = {Public Space in Late Antique Emerita},
  journaltitle = AJA,
  shortjournal = AJA-short,
  volume       = {120},
  pages        = {67--97},
  year         = {2016},
  jstor        = {10.3764/aja.120.1.0067},
  number       = {1},
  zenon        = {001454110},
}
\end{lstlisting}

looks like

\printbib[5em]{Osland2016}

and with |bibfullname| it will change to:

\begin{bibbsp}{Osland2016}
\parbox[t]{1.7cm}{Osland 2016} \parbox[t]{9.4cm}{{\color{red}Daniel} Osland, Abuse or Reuse? Public Space in Late Antique Emerita, AJA 120/ 1, 2016, 67–97,\\
...}
\end{bibbsp}

\subsubsection{noabbrv}\label{noabbrevs}
\DescribeMacro{noabbrv}
According to the guidelines of the German Archaeological Institute journal titles and series have to be abbreviated.
Therefore the fields |shortjournal| or |shortseries| will be considered. 
If you like to have printed full names of journals and series instead you can switch on the option |noabbrv|.
For an example see \cref{string,Koyunlu1990}.

\subsubsection{publisher}\label{publisher}
\DescribeMacro{publisher} 
Once activated all locations and the publisher are printed. 
This will lead to a different output of the edition which will be right in front of the year.
In case of reprint or second edition the first edition |origyear| will be put in square brackets after the year.

\begin{lstlisting}[style=bibentry,label=Emme2013,caption={{@}Book\{Emme2013,…\} }]
@Book{Emme2013,
  author    = {Burkhard Emme},
  title     = {Peristyl und Polis},
  subtitle  = {Entwicklung und Funktionen öffentlicher griechischer Hofanlagen},
  publisher = WdG,
  location  = Berlin #{ and New York}, %@String partly used
  year      = {2013},
  series    = {Urban Spaces},
  number    = {1},
}
\end{lstlisting}
\begin{refsection}\end{refsection}

Default settings produce:
\printbib[5em]{Emme2013}
 
By activating option |publisher| you obtain:

\begin{bibbsp}{Emme2013}
\parbox[t]{1.7cm}{Emme 2013} \parbox[t]{9.4cm}{B. Emme, Peristyl und Polis. Entwicklung und Funktionen öffentlicher griechischer Hofanlagen, Urban Spaces 1 (Berlin {\color{red} – New York: Walter de Gruyter} 2013)}
\end{bibbsp}
 
And here a more detailed example with |origlocation|, |origyear| and |origpublisher|:
\begin{lstlisting}[style=bibentry,label=Neufert2002,caption={{@}Book\{Neufert2002,…\} }]
@Book{Neufert2002,
  author       = {Neufert, Ernst},
  editor       = {Neufert, Peter and Neufert, Cornelius and Neff, Ludwig and Franken, Corinna},
  title        = {Bauentwurfslehre},
  subtitle     = {Grundlagen, Normen, Vorschriften ...},
  publisher    = VT, %@String used
  origpublisher = {Mann},
  location     = {Wiesbaden},
  year         = {2002},
  edition      = {37},
  origlocation = Berlin, %@String used
  origyear     = {1936},
}
\end{lstlisting}
 
\printbib[5em]{Neufert2002}

\begin{bibbsp}{Neufert2002}
\parbox[t]{1.7cm}{Neufert  2002} \parbox[t]{9.4cm}{%
E. Neufert, Bauentwurfslehre. Grundlagen, Normen, Vorschriften ... Ed. by P.~Neufert – C. Neufert – L. Neff  – C. Franken {\color{red} (Wiesbaden: Vieweg \textsuperscript{3}2002 [Berlin: Mann 1936])}}
\end{bibbsp}
 
\subsubsection{edby}\label{edby}
\DescribeMacro{edby}
This option gives you a different output and position of editors: 
Instead of embedding ›(ed.)‹/›(Hrsg.)‹ right after the editor name
 it places the adjunct ›ed. by‹/›hrsg. v.‹ behind the editor. 
 Furthermore, in case of |@Incollections| and |@Inproceedings| editor names and book title switch their positions as it is shown below.

\begin{lstlisting}[style=bibentry,label=Wulf-Rheidt2013,caption={{@}Inproceedings\{Wulf-Rheidt2013,…\} }]
@Inproceedings{Wulf-Rheidt2013,
  author       = {Wulf-Rheidt, Ulrike},
  title        = {Der Palast auf dem Palatin -- Zentrum im Zentrum},
  subtitle     = {Geplanter Herrschersitz oder Produkt eines langen Entwicklungsprozesses?},
  pages        = {277--289},
  editor       = {Dally, Ortwin and Fless, Friederike and Haensch, Rudolf and Pirson, Felix and Sievers, Susanne},
  booktitle    = {Politische Räume in vormodernen Gesellschaften},
  booksubtitle = {Gestaltung – Wahrnehmung – Funktion},
  location     = {Rahden/Westf\adddot},
  publisher    = VML,		%@String used
  year         = {2013},
  venue        = Berlin, 	%@String used
  eventdate    = {2009-11-18/2009-11-22},
  eventtitle   = {Internationale Tagung des DAI und des DFG-Exzellenzclusters TOPOI},
  zenon       = {001371402},
  number       = {6},
  series       = MKT,		%@String used
  shortseries  = MKT-short,	%@String used
}
\end{lstlisting}

Without option |edby| this |@Inproceedings| would look like:
 
\printbib[7em]{Wulf-Rheidt2013}

But activating |edby| it changes to:

\begin{bibbsp}{Wulf-Rheidt2013}
\parbox[t]{2.3cm}{Wulf-Rheidt 2013} \parbox[t]{9cm}{%
U. Wulf-Rheidt, Der Palast auf dem Palatin – Zentrum im Zentrum. Geplanter Herrschersitz oder Produkt eines langen Entwicklungsprozesses?, in:  {\color{red}Politische Räume in vormodernen Gesellschaften. Gestaltung – Wahrnehmung – Funktion, ed. by O. Dally – F. Fless – R. Haensch – F. Pirson – S. Sievers}. Internationale Tagung des DAI und des DFG-Exzellenzclusters TOPOI Berlin November 18–22, 2009, MKT 6 (Rahden/Westf. 2013) 277–289}
\end{bibbsp}
 
 
\subsubsection{width}\label{width}
\DescribeMacro{width}
|width| controls the width between label (which consists usually of |author| and |year|) and reference in the bibliography, pre-defined as |4em|.
If you wish to have it bigger or smaller you can change it to every length you would like to have:

|width =| \meta{length}

\meta{length} stands for the length you want (e.\,g. |3em|, |7pt| or |4cm|), you can even do |-1em|; 
then there is no indent at all.

\subsubsection{counter}\label{counter}
\DescribeMacro{counter} 
If you like to know how many times you cited an author or work then use this option called |counter|.\footnote{The idea is based on \href{http://tex.stackexchange.com/a/14159/98739}{tex.stackexchange.com/a/14159/98739} and has been modified.} 
Depending on the language you chose in the preamble of your document the information will be given in German (|ngerman|) or in English (if not |ngerman|).

\begin{bibbsp}{Boehm2001}
\parbox[t]{3cm}{Böhm – Eickstedt 2001} \parbox[t]{8cm}{%
S. Böhm – K.-V. v. Eickstedt (Hrsg.), Ithake. Festschrift Jörg Schäfer (Würzburg 2001)  $\vert$  {\scshape  wurde 1-mal zitiert.}}
\end{bibbsp}

If there has been no citation in the text (but maybe a \cs{citeauthor} or \cs{citetitle}):
\begin{bibbsp}{Boehm2001}
\parbox[t]{3cm}{Böhm – Eickstedt 2001} \parbox[t]{8cm}{%
S. Böhm – K.-V. v. Eickstedt (Hrsg.), Ithake. Festschrift Jörg Schäfer (Würzburg 2001)  $\vert$  {\scshape  wurde {\color{red}{keinmal}} zitiert.}}
\end{bibbsp} 

For all languages besides German:
\begin{bibbsp}{Boehm2001}
\parbox[t]{3cm}{Böhm – Eickstedt 2001} \parbox[t]{8cm}{%
S. Böhm – K.-V. v. Eickstedt (ed.), Ithake. Festschrift Jörg Schäfer (Würzburg 2001) $\vert$  {\scshape cited {{\color{red}{not once}}}.}}
\end{bibbsp}
  
If there has been only one citation:
\begin{bibbsp}{Boehm2001}
\parbox[t]{3cm}{Böhm – Eickstedt 2001} \parbox[t]{8cm}{%
S. Böhm – K.-V. v. Eickstedt (ed.), Ithake. Festschrift Jörg Schäfer (Würzburg 2001) $\vert$  {\scshape cited 1 time.}}
\end{bibbsp}

If there has been more than one citations:
\begin{bibbsp}{Boehm2001}
\parbox[t]{3cm}{Böhm – Eickstedt 2001} \parbox[t]{8cm}{%
S. Böhm – K.-V. v. Eickstedt (ed.), Ithake. Festschrift Jörg Schäfer (Würzburg 2001) $\vert$  {\scshape cited 3 times.}}
\end{bibbsp}

Note that |biblatex| provides a related option |backref| which lists per reference every page 
that contains the cited reference. But having a different goal that option doesn't support counts. 
 
\subsubsection{initials}\label{initials}
\DescribeMacro{counter} 
First names are abbreviated keeping digraphs and trigraphs instead of simple first letter initials.\footnote{Option was included from \href{http://tex.stackexchange.com/a/295486/98739}{tex.stackexchange.com/a/295486/98739}}

\subsection{Bibliography entry options}\label{options-bibentry}
\subsubsection{ancient}\label{ancient}
\DescribeMacro{ancient}
This option was found in an excellent |bib|\LaTeX-style called  |geschichtsfrkl| (v.1.1 by Jonathan Zachhuber), 
so after some modifications we adopted and included it into |nottsclassic|.
 
If you intend to cite ancient authors we strongly recommend this option to you 
because it enables you to cite ancient texts in the style archaeologists and 
historians are used to \emph{plus} to get entirely supported bibliography referencing.

Let us have a look at an example:
\begin{lstlisting}[style=bibentry,label=Cic:Att,caption={{@}Book\{Cic:Att,…\} }]
@Book{Cic:Att,
  author       = {Tullius Cicero, Marcus},
  editor       = {Kasten, Helmut},
  title        = {Atticus-Briefe},
  publisher    = AW, 	%@String used
  location     = {Düsseldorf and Zürich},
  year         = {1980},
  series       = {Tusculum Bücherei},
  edition      = {3},
  origyear     = {1959},
  origtitle    = {epistulae ad Atticum},
  origlanguage = {latin},
  translator   = {Kasten, Helmut},
  shorthand    = {Cic. Att.},
  shortauthor  = {Cicero},
  keywords     = {ancient},
  options      = {ancient}, %!!
}
\end{lstlisting}

Instead of applying |author| and |year| as label the option |ancient| takes the field |shorthand| into account.

So you write in your text
\begin{lstlisting}
\cite[1, 3,3]{Cic:Att}\end{lstlisting} 

and it will be printed as

\begin{refsection}
\begin{bsp}
\cite[1, 3,3]{Cic:Att}
\end{bsp}\end{refsection}

Equally the field |shorthand| is used as label in the bibliography instead of an author-year label:
\printbib{Cic:Att}

Take notice how |options={ancient}| treats editor and translator in this example. 
While the ancient author is mentioned first, translator and editor are put behind the title, 
in this case even united due to fact that editor and translator are identical.

This works not only with |@Book| but also with those ancient texts which are part of an |@Incollection| employing them similarly to the entries defined as |@Book|.

Another example:
\begin{lstlisting}[style=bibentry,label=Cic:Sest,caption={{@}Inbook\{Cic:Sest,…\} }]
@Inbook{Cic:Sest,
  author       = {Tullius Cicero, Marcus},
  title        = {Rede für P.\ Sestius},
  booktitle    = {Die politischen Reden},
  year         = {1993},
  editor       = {Fuhrmann, Manfred},
  volume       = {II},
  publisher    = AW,		%@String used
  pages        = {110--185},
  origlanguage = {latin},
  series       = {Sammlung Tusculum},
  location     = Munich,			%@String used
  intranslator = {Fuhrmann, Manfred},
  keywords     = {ancient},
  options      = {ancient},
  origtitle    = {pro P. Sestio},
  shortauthor  = {Cicero},
  shorthand    = {Cic. Sest.},
}
\end{lstlisting}

Even support for several languages is taken into account as this bibliography entries show: 
\printbiball{Cic:Sest}

\subsubsection{frgancient}\label{frgancient}
\DescribeMacro{frgancient}
When dealing with fragments of ancient texts often the edition respective the editor is important to cite. 
If you want to cite such a collection of fragments use |options={frgancient}|.
In this case the editor (|shorteditor| or |editor|) will be put in after the \marg{postnote}. 

\begin{lstlisting}[style=bibentry,label=Fest,caption={{@}Book\{Fest,…\} }]
@Book{Fest,
  author      = {Pompeius Festus, {\relax Sex}tus},
  editor      = {Lindsay, Wallace Martin},
  title       = {De verborum significatu quae supersunt cum Pauli epitome},
  publisher   = {Teubner},
  location    = Leipzig, 		%@String used
  year        = {1965},
  series      = {Bibliotheca scriptorum et Grecorum et Romanorum Teubnerina},
  origyear    = {1913},
  shorthand   = {Fest.},
  shortauthor = {Festus},
  keywords    = {ancient},
  options     = {frgancient},
  shorteditor = {L},
}
\end{lstlisting}

When you cite this entry in this example with |\cite[3]{Fest}| the field  |shorteditor| will be shown.
\begin{bsp}
Fest. 3 L
\end{bsp}

In the bibliography the reference differentiates slightly from |options = {ancient}| because of the missing ancient author of fragment collections, 
so the editors name is printed in the first place:
\printbib[2em]{Fest}

\subsubsection{corpus}\label{corpus}
\DescribeMacro{corpus}
There are some corpora (e.\,g. inscriptions, minted coins, vases, etc.) which are usually cited with a common abbreviation. 
Those abbreviations are typed in in the field |shorthand|.
To cite such a corpus with its abbreviation you have to write an additional |options={corpus}| in the bibliographical entry.
Now you can cite it as usual with \oarg{prenote} or \oarg{postnote} you like to have.

This example shows the behaviour with a corpus that is common for Latin epigraphy:
\begin{lstlisting}[style=bibentry,label=CIL,caption={{@}Book\{CIL,…\} }]
@Book{CIL,
  title     = CIL, %@String used
  location  = Berlin, %@String used
  year      = {1863--},
  shorthand = CIL-short,	%@String used
  keywords  = {corpus}, %!
  options   = {corpus},
}
\end{lstlisting}

It will be cited with:
\begin{lstlisting}
\cite[06, 01234]{CIL}.
\end{lstlisting}

And the result is:
\begin{bsp}
CIL 06, 01234
\end{bsp}

The field |shorthand| will be used for bibliography reference,
where it is listed as label.
\printbib{CIL}

If you also set something like |keywords={corpus}| then you can make a separate bibliography with all the copora cf. \cref{bibliographie}.

\changes{v1.1}{2015/06/15}{Modifikation der Option |corpus|.}


\section{Examples of entry types}\label{beispiele}
The style |nottsclassic| defines several so-called bibliographical drivers,
which allow to cite different kind of works.
Below we describe how they are working and which fields you should fill out.
For the examples we don't use any further options which are described above.

\subsection{Type \texttt{@Book}}\label{book}
\DescribeMacro{@Book}
\DescribeMacro{@Collection}%\footnote{Der Typ |@collection| entspricht hier dem Typ |@Book|.}
Let’s start with an easy example: 
a book or a collection (both are treated equally).
You can use the following fields:
\begin{description}
\item[mandatory:] 
|author|/|editor|, 
|title|, |subtitle|, |titleaddon|,
|location|, |year|,
\item[optional:]
|maintitle|, |mainsubtitle|, |maintitleaddon|, |volume|, 
|publisher|, |series|, |number|, |edition|, 
|origyear|, |origlocation|, |origpublisher|, 
|translator|, |origlanguage|,
|related|, |relatedtype|,
|doi|, |url|, |urldate|, |eprint|, |eprinttype|, |note|, |pubstate|, 
 \end{description}
 
 
An entry of a book might look as this in your |bib|-file:
\begin{lstlisting}[style=bibentry,label=Mann2011,caption={{@}Book\{Mann2011,…\} }]
@Book{Mann2011,
  author    = {Mann, Christian},
  title     = {\enquote{Um keinen Kranz, um das Leben kämpfen wir!}},
  subtitle  = {Gladiatoren im Osten des Römischen Reiches und die Frage der Romanisierung},
  publisher = {Verlag Antike},
  location  = Berlin, 	%@String used
  year      = {2011},
  series    = {Studien zur Alten Geschichte},
  number    = {14},
}
\end{lstlisting}

A citation in a footnote is done with
\begin{lstlisting}
\footnote{\cite[Vgl.][142--144]{Mann2011}.}
\end{lstlisting} 

and it looks like this in footnotes\footnote{Vgl. Mann 2011, 142--144.}
and like this in the bibliography.
\printbib[5em]{Mann2011}

\subsubsection{›Festschrift‹, commemorative volume, catalogue, etc.}
To mark that the book or collection is a so-called ›Festschrift‹/ commemorative volume, 
or an exhibition or auction catalogue you need an additional note to make it clear.
We suggest to use the field |titleaddon| or if it is a |@Incollection| or |@Inproceedings| you can use the field |booktitleaddon| (for papers in collections see \cref{inbook}).
\begin{lstlisting}[style=bibentry,label=Boehm2001,caption={{@}Book\{Boehm2001,…\} }]
@Book{Boehm2001,
  editor     = {Böhm, Stephanie and Eickstedt, Klaus-Valtin von},
  title      = {Ithake},
  publisher  = {Ergon-Verlag},
  location   = {Würzburg},
  year       = {2001},
  titleaddon = {Festschrift Jörg Schäfer},
}
\end{lstlisting}

\printbib[9em]{Boehm2001}
 
\subsubsection{Translated book}
If you cite a translated book you can link it to the original book and let display the translator as well as the original language. 
To obtain this fill the fields |related| and |relatedtype| (for further information see the informations about reviews in \cref{review}).

It will be more clear with this example:
The first edition of \citetitle*{Zanker2009} by \citeauthor*{Zanker2009} has been published in 1987 (|origyear|), but by now it has been released in its 5th edition.

\begin{lstlisting}[style=bibentry,label=Zanker2009,caption={{@}Book\{Zanker2009,…\} }]
@Book{Zanker2009,
  author        = {Zanker, Paul},
  title         = {Augustus und die Macht der Bilder},
  publisher     = CHB, 	%@String used
  location      = Munich,	%@String used
  year          = {2009},
  edition       = {5},
  origlocation  = Leipzig, %@String used
  origyear      = {1987},
  eprint        = {000250713},
  eprinttype    = {zenon},
  language      = {german},
  origpublisher = {Koehler \& Amelang},
}
\end{lstlisting}

In \citeyear{Zanker1988}, one year later after the first edition (|origyear|) the book has been translated by A. H. Shapiro:
\begin{lstlisting}[style=bibentry,label=Zanker1988,caption={{@}Book\{Zanker1988,…\} }]
@Book{Zanker1988,
  author      = {Zanker, Paul},
  title       = {The Power of Images in the Age of Augustus},
  publisher   = UMP,		%@String used
  location    = {Ann Arbor},
  year        = {1988},
  series      = {Jerome Lectures},
  number      = {16},
  translator  = {Shapiro, Alan H.},
  language    = {english},
  related     = {Zanker2009},
  relatedtype = {translationof},
}
\end{lstlisting}

The translated book |Zanker1988| is connected with the book |Zanker2009| via the field |related = |\marg{bibtex-key} 
and the relation got specified with the field |relatedtype|, in this case it is a translation so |={translationof}|.

You don't have to cite |Zanker2009| to have the information visible in the bibliography. 
It will be included automatically. For the following bibliography we only use |\cite{Zanker1988}|:

\printbiball[5em]{Zanker1988}


\subsubsection{Multiple volumes of a monograph}
There is the case that you have to cite a book which consists of several volumes, 
usually there is a volume with text and one volume with plates.
To cite particularly e.\,g. the second volume you can do the following.
Let’s assume this is the bibliography entry:
\begin{lstlisting}[style=bibentry,label=MacDonald1986,caption={{@}Book\{MacDonald1986,…\} }]
@Book{MacDonald1986,
  author    = {MacDonald, William L.},
  title     = {An urban Appraisal},
  publisher = YUP,		%@String used
  location  = {New Haven and }# London, %@String used
  year      = {1986},
  maintitle = {The Architecture of the Roman Empire},
  volume    = {II},
  series    = {Yale Publications in the History of Art},
  number    = {35},
}
\end{lstlisting}
In the bibliography the main title of the monograph (|maintitle|)
and the title of the book (|title|) are shown separately  so the volume  (|volume|) 
appears before the title of the book
\printbib[7em]{MacDonald1986}


\subsection{Type \texttt{@Inbook / @Incollection}}\label{inbook}
\DescribeMacro{@Incollection}\DescribeMacro{@Inbook}
Single entries/chapters of a collection are cited best when they are set up as  |@Incollection| or |@Inbook|.

\begin{description}
\item[mandatory:] 
|author|, |title|, |subtitle|, |titleaddon|,
|editor|,  |booktitle|, |booksubtitle|, |booktitleaddon|,
|location|, |year|, |pages|, 
\item[optional:]
|maintitle|, |mainsubtitle|, |maintitleaddon|, |volume|, 
|publisher|, |series|, |number|, |edition|, 
|origyear|, |origlocation|, |origpublisher|, 
|translator|, |origlanguage|,
|related|, |relatedtype|,
|doi|, |url|, |urldate|, |eprint|, |eprinttype|, |note|, |pubstate|, 
 \end{description}
 
 
 
This following example should make it clear:
 \begin{lstlisting}[style=bibentry,label=Carter2014,caption={{@}Incollection\{Carter2014,…\} }]
@Incollection{Carter2014,
  author    = {Carter, Michael J. and Edmondson, Jonathan},
  title     = {Spectacle in Rome, Italy, and the Provinces},
  pages     = {537--558},
  editor    = {Bruun, Christer and Edmondson, Jonathan},
  booktitle = {The Oxford Handbook of Roman Epigraphy},
  publisher = OUP,		%@String used
  location  = {Oxford},
  year      = {2014},
}
\end{lstlisting}

\printbib[11em]{Carter2014}

You can also have contributions to a ›Festschrift‹ etc. set up as |@Incollection|,
but then notice the additional information in |booktitleaddon|.
\begin{lstlisting}[style=bibentry,label=Hoelscher2001,caption={{@}Incollection\{Hoelscher2001,…\} }]
@Incollection{Hoelscher2001,
  author         = {Hölscher, Tonio},
  title          = {Schatzhäuser -- Banketthäuser?},
  pages          = {143--152},
  editor         = {Böhm, Stephanie and Eickstedt, Klaus-Valtin von},
  booktitle      = {Ithake},
  publisher      = {Ergon-Verlag},
  location       = {Würzburg},
  year           = {2001},
  booktitleaddon = {Festschrift Jörg Schäfer},
}
\end{lstlisting}
In the bibliography it will look like:
\printbib[6em]{Hoelscher2001}

\subsubsection{Short series}
Some books or collections are part of a small series (not an ongoing series).
This book is part of the series called abbreviated \emph{MemAmAc}.
Have a look:
\begin{lstlisting}[style=bibentry,label=Fentress2003,caption={{@}Incollection\{Fentress2003,…\} }]
@Incollection{Fentress2003,
  author       = {Fentress, Elizabeth and John Bodel and Adam Rabinowitz and Rabun Taylor},
  title        = {Cosa in the Republic and Early Empire},
  pages        = {13--62},
  editor       = {Fentress, Elizabeth},
  booktitle    = {An Intermittent Town},
  booksubtitle = {Excavations 1991--1997},
  publisher    = UMP,		%@String used
  location     = {Ann Arbor, Mich.},
  year         = {2003},
  volume       = {V},
  series       = MemAmAc,		%@String used
  number       = {2},
  maintitle    = {Cosa},
  shortseries  = MemAmAc-short,		%@String used
}
\end{lstlisting}
As we can see it is the fifth volume (|volume|) of the series with the main title 
\emph{Cosa} (|maintitle|) but has an individual title (|title|) which is
\emph{Cosa in the Republic and Early Empire}, furthermore it is the second book (|number|) 
of the series  \emph{MemAmAc} (|series|).

Notice the different language-based behaviour of ›et. al.‹ for more than two authors/editors.
\printbiball[7em]{Fentress2003}

 
\subsubsection{Inventory catalogue}
The output of an inventory catalogue changes slightly compared to collections or sth. similar. 
The title is omitted and therefore there is no comma after the author’s name.
We provide two examples so you see the difference.
\begin{lstlisting}[style=bibentry,label=Kohlmeyer1983,caption={{@}Inbook\{Kohlmeyer1983,…\} }]
@Inbook{Kohlmeyer1983,
  author       = {K. Kohlmeyer},
  booktitle    = {Tierbilder aus vier Jahrtausenden},
  year         = {1983},
  editor       = {U. Gehrig},
  booksubtitle = {Antiken der Sammlung Mildenberg},
  pages        = {20 Nr. 9},
  location     = Mainz, %@String used
}
\end{lstlisting}
and the second example

\begin{lstlisting}[style=bibentry,label=Parlasca1969,caption={{@}Inbook\{Parlasca1969,…\} }]
@Inbook{Parlasca1969,
  author    = {K. Parlasca},
  booktitle = {Helbig},
  year      = {1969},
  volume    = {III},
  edition   = {4},
  pages     = {98\psq\ Nr. 2176},
  location  = Tuebingen, %@String used
}
\end{lstlisting}


\printbib[6em]{Kohlmeyer1983}
\begin{refsection}\end{refsection}
\printbiball[6em]{Parlasca1969}


\subsection{Type \texttt{@Article}}\label{article}
\DescribeMacro{@Article} This is probably the most common type 
because you find detailed information about specific topics in articles.

\begin{description}
\item[mandatory:] 
|author|, |title|, |subtitle|, |titleaddon|,
|journaltitle|, |shortjournal|, |volume|, |number|, |issue|
|year|, |pages|, 
\item[optional:]
|translator|, |origlanguage|,
|related|, |relatedtype|,
|doi|, |url|, |urldate|, |eprint|, |eprinttype|, |note|, |pubstate|, 
 \end{description}

Here we have an example which will explain the (required)  fields:
\begin{lstlisting}[style=bibentry,label=Evangelidis2014,caption={{@}Article\{Evangelidis2014,…\} }]
@Article{Evangelidis2014,
  author       = {Evangelidis, Vasilis},
  title        = {Agoras {and} Fora},
  subtitle     = {Developments in the Central Public Space of the Cities of Greece during the {Roman} Period},
  journaltitle = BSA,		%@String used
  shortjournal = BSA-short,		%@String used
  volume       = {109},
  pages        = {335--356},
  year         = {2014},
  doi          = {10.1017/s006824541400015x},
}
\end{lstlisting}
In line 5 and 6 you can also write the full or abbreviated journal title in the fields |journaltitle| or |shorttitle| (e.\,g. |journaltitle = {British School of Athens}|, |shortjournal = {BSA}|), but we chose to use a |@String| (cf. \cref{list-bibancient,string}) again.
\printbib[7em]{Evangelidis2014}


\subsection{Type \texttt{@Proceedings}}\label{proceedings}
\DescribeMacro{@Proceedings}
Similar to a collection but still different in the bibliographical output is proceedings.
Therefore we recommend to use the type |@Proceedings|.
The difference lies in the additional mandatory fields |venue|, |eventdate| and |eventtitle|. 
Everything else is like the type |@Book|.

\begin{description}
\item[mandatory:] 
%|author|/
|editor|, 
|title|, |subtitle|, |titleaddon|,
|venue|, |eventdate|, |eventtitle|,
|location|, |year|
\item[optional:]
|maintitle|, |mainsubtitle|, |maintitleaddon|, |volume|, 
|publisher|, |series|, |number|, |edition|, 
|origyear|, |origlocation|, |origpublisher|, 
|translator|, |origlanguage|,
|related|, |relatedtype|,
|doi|, |url|, |urldate|, |eprint|, |eprinttype|, |note|, |pubstate|, 
 \end{description}
 
An example:
 \begin{lstlisting}[style=bibentry,label=Kurapkat2014,caption={{@}Proceedings\{Kurapkat2014,…\} }]
@Proceedings{Kurapkat2014,
  title        = {Die Architektur des Weges},
  year         = {2014},
  editor       = {Kurapkat, Dietmar and Schneider, Peter I. and Wulf-Rheidt, Ulrike},
  subtitle     = {Gestaltete Bewegung im gebauten Raum},
  eventtitle   = {Kolloquium Architekturreferat des DAI},
  eventdate    = {2012-02-08/2012-02-11},
  venue        = Berlin, 		%@String used
  series       = DiskAB,		%@String used
  number       = {11},
  organization = {Architekturreferat des DAI},
  publisher    = {Schnell + Steiner},
  location     = Regensburg, 		%@String used
  shortseries  = DiskAB-short,		%@String used
}
\end{lstlisting}
With |venue| we specify the place where the proceeding took place 
(e.\,g. \emph{Berlin} -- |location| is where the book was printed and is connected 
to the |publisher| e.\,g. \emph{Regensburg}),
|eventtitle| is used for a special title of the proceeding (e.\,g. \emph{Kolloquium Architekturreferat des DAI}),
|eventtitle| gives the date(range) when the proceeding was hold and has to be typed in in the format YYYY-MM-DD, 
a range has to be separated with a |/| (e.\,g.  \emph{2012-02-08/2012-02-11}).

In the bibliography the information of these additional fields will be used (of course) as this, notice how the output of the date changes according to the chosen language.
\printbiball[7em]{Kurapkat2014}

\subsection{Type \texttt{@Inproceedings}}\label{inproceedings}
\DescribeMacro{@Inproceedings}
This entry type works similar to |@Procecedings| and |@Incollection| and therefore the needed fields are straightforward to use:

\begin{description}
\item[mandatory:] 
|author|, |title|, |subtitle|, |titleaddon|,
|editor|,  |booktitle|, |booksubtitle|, |booktitleaddon|,
|venue|, |eventdate|, |eventtitle|,
|location|, |year|, |pages|, 
\item[optional:]
|maintitle|, |mainsubtitle|, |maintitleaddon|, |volume|, 
|publisher|, |series|, |number|, |edition|, 
|origyear|, |origlocation|, |origpublisher|, 
|translator|, |origlanguage|,
|related|, |relatedtype|,
|doi|, |url|, |urldate|, |eprint|, |eprinttype|, |note|, |pubstate|, 
\end{description}
 
 
 
 \begin{lstlisting}[style=bibentry,label=Torelli1991,caption={{@}Inproceedings\{Torelli1991,…\} }]
@Inproceedings{Torelli1991,
  author     = {Torelli, Mario},
  title      = {Il \enquote{diribitorium} di Alba Fucens e il \enquote{campus} eroico di Herdonia},
  pages      = {39--63},
  editor     = {Mertens, Josef},
  booktitle  = {Comunitá indigene e problemi della romanizzazione nell’Italia centro\--meri\-dionale (IV--III sec. a.C.)},
  location   = Brussels, 		%@String used
  publisher  = {Institut Historique Belge de Rome},
  year       = {1991},
  venue      = Rome #{, Academia Belgica},		%@String used
  eventdate  = {1990-02-01/1990-02-03},
  eventtitle = {Actes du Colloque International Organisé à l'Occasion du 50. Anniversaire de l'Academia Belgica et du 40. Anniversaire des Fouilles Belges en Italie},
  hyphenate  = {italian},
  language   = {italian},
  number     = {29},
  series     = {Études de philologie, d'archéologie et d'histoire anciennes},
  shorttitle = {Il \enquote{diribitorium}},
}
\end{lstlisting}
It will be printed as:
 
\printbiball[4em]{Torelli1991}

 \subsection{Type \texttt{@Reference}}\label{reference}
 \DescribeMacro{@Reference}
 This entry type can be used especially for references if you want to cite it as whole or if you need to relate to it. 
We provide an example below---cf. \cref{inreference}

You don’t need to fill out many fields to have a working entry:
\begin{description}
\item[mandatory:] |title|, |shorthand|,
\item[optional:] 
 |editor|, |subtitle|, |titleaddon|,
 |location|, |year|
|maintitle|, |mainsubtitle|, |maintitleaddon|,
|related|, |relatedtype|,
|publisher|, |series|, |number|, |edition|, |volume|,
|doi|, |url|, |urldate|, |eprint|, |eprinttype|, |note|, |pubstate|, 
\end{description}

And so a complete entry is quite small:
\begin{lstlisting}[style=bibentry,label=LIMC,caption={{@}Reference\{LIMC,…\} }]
@Reference{LIMC,
  title     = LIMC,
  keywords  = {corpus},
  options   = {corpus},
  shorthand = LIMC-short,
}
\end{lstlisting}
 
But you can also have it  more detailed  like this one:
\begin{lstlisting}[style=bibentry,label=Lexikon-der-Technik,caption={{@}Reference\{Lexikon-der-Technik,…\} }]
@Reference{Lexikon-der-Technik,
  editor    = {Otto Lueger},
  title     = {Lexikon der gesamten Technik und ihrer Hilfswissenschaften},
  year      = {1904--1920},
  edition   = {2},
  location  = Stuttgart, 	%@String used
  keywords  = {corpus},
  shorthand = {Lexikon d. T.},
}
\end{lstlisting}

 \subsection{Type \texttt{@Inreference}}\label{inreference}
 \DescribeMacro{@Inreference}
Besides a whole reference you can also -- and which is more likely -- cite only an entry of it via the type  |@Inreference|.

It will be more clear with an example:
  \begin{lstlisting}[style=bibentry,label=Neils1994,caption={{@}Inreference\{Neils1994,…\} }]
@Inreference{Neils1994,
  author    = {Neils, Jenifer},
  title     = {Theseus},
  booktitle = LIMC-short,		%@String used
  pages     = {922--951},
  year      = {1994},
  volume    = {7.1},
  keywords  = {lexikon},
}
\end{lstlisting}
You can cite this entry with any of the provided \cs{cite}-commands above---cf. \cref{cite-commands,faq:inreference}.

But for the final display of the entry  you have two possibilities:
 \begin{enumerate}
\item\label{inreference:a} 

Either you want it in the default style ›author-year‹, so it will have a label and is referenced 
to your final bibliography, then you don’t have to do anything.
In the bibliography it will look like
\printbib{Neils1994}

\item\label{inreference:b} 
The German Archaeological Institute has a special rule for inreferences in the footnote.
Then the output will be like:\\
|reference volume (year) pages s.v. title (author)| \\
and there will be no reference to the bibliography since the entry is fully described in the footnote.
\DescribeMacro{inreferences} If you prefer this method you have to use the preamble option called  |inreferences|---cf. \cref{inreferences}
Then it will look like this in the footnote:
\begin{bsp}
LIMC 7.1 (1994) 922--951 s. v. Theseus (J. Neils)
\end{bsp}
When you have the \oarg{postnote} filled out in a citation which belongs to an |@Inreference| then it won’t be printed in the end of the citation.
\begin{lstlisting}
\cite[vgl.][930 Nr. 283]{Neils1994}.
\end{lstlisting}
The |postnote| \oarg{930 Nr. 283} will be printed instead of the |pages|:
 \begin{bsp}
vgl. LIMC 7.1 (1994) 930 Nr. 283 s. v. Theseus (J. Neils).
 \end{bsp}
 
\end{enumerate}

As mentioned above it is of advantage when you relate entries like  an |@Inreference| with its |@Reference|. 
And since not all references have a ›canonical‹ abbreviation (e.\,g. RE, LIMC, DNP, LTUR, LÄ, etc.) it might be necessary to define a |shorthand|.
This is shown in the example below.
 
\begin{lstlisting}[style=bibentry,label=Weinbrenner1914,caption={{@}Inreference\{Weinbrenner1914,…\} }]
@Inreference{Weinbrenner1914,
  author    = {Weinbrenner},
  title     = {Rennbahn},
  booktitle = {Lexikon d. T.},
  pages     = {636--637},
  year      = {1914},
  related   = {Lexikon-der-Technik},
  volume    = {9},
  number    = {2},
}
\end{lstlisting}
As you see this entry is related to |Lexikon-der-Technik| which is described above and has a |shorthand = {Lexikon d. T.}|
You just need to make sure that  |booktitle| of the |@Inreference| and the |shorthand| of 
the  |@Reference| are equal so the title can be referenced properly in the bibliography.

For the following bibliography result we just typed (assuming that the entries were already cited in text):
\begin{lstlisting}
\printbibliography[keyword=corpus,title={Corpora}]
\printbibliography[notkeyword=corpus]
\end{lstlisting}

\begin{refsection}
\begin{bibbsp}{Lexikon-der-Technik,Weinbrenner1914}
\nocite{Lexikon-der-Technik,Weinbrenner1914}
\printbibliography[keyword=corpus,title={Corpora}]
\printbibliography[notkeyword=corpus]
\end{bibbsp}
\end{refsection}


 
\subsection{Type \texttt{@Review}}\label{review}
\DescribeMacro{@Review}
Reviews in journals are best cited when they are edited as a |@Review|.
For a full citation of a review you have to name the reviewed work in detail.
The following example will show an easy way to combine the review with the reviewed work.

\begin{description}
\item[mandatory:] 
|author|, |title|, |subtitle|, |titleaddon|,
|journaltitle|, |shortjournal|, |volume|, |number|, |issue|
|year|, |pages|, 
|related|, |relatedtype|,
\item[optional:]
|translator|, |origlanguage|,
|doi|, |url|, |urldate|, |eprint|, |eprinttype|, |note|, |pubstate|, 
 \end{description}

What you need are two separate entries: one as a |@Review| the other is a |@Book|, |@Collection|, |@Proceedings| or something else.

First we have the reviewed work:
\begin{lstlisting}[style=bibentry,label=Welch2007,caption={{@}Book\{Welch2007,…\} }]
@Book{Welch2007,
  author    = {Welch, Katherine E.},
  title     = {The {Roman} Amphitheatre},
  subtitle  = {From its Origins to the Colosseum},
  publisher = CUP,		%@String used
  location  = {Cambridge and New York},
  year      = {2007},
}
\end{lstlisting}
then the review itself:
\begin{lstlisting}[style=bibentry,label=Bell2011,caption={{@}Review\{Bell2011,…\} }]
@Review{Bell2011,
  author       = {Bell, Sinclair},
  number       = {1},
  pages        = {1--4},
  volume       = {115},
  journaltitle = AJA,		%@String used
  shortjournal = AJA-short,		%@String used
  related      = {Welch2007},
  relatedtype  = {reviewof},
  year         = {2011},
  publisher    = {Archaeological Institute of America},
}
\end{lstlisting}
You maybe noticed that the reveiw (|Bell2011|) is with the field |related| in line 8 connected to the entry |Welch2007|.
In addition we not only need a connected work but also to qualify the relation:
This is done in line 9 with |relatedtype = {reviewof}|.
This so-called |bibstring| is especially for reviews and contains the language based correct abbreviation for \emph{Review of} or e.\,g. \emph{Rez. zu}.
You don’t have to type in all relevant information of the reviewed work in the entry of the review, 
since they will be inserted automatically and dynamically with the  |related|-function. 
So whenever settings in the reviewed work are changed the print of the review will be automatically adjusted. 
Furthermore, even if the review is cited, the reviewed work won't be listed in the bibliography until it is explicitly cited in the text.
\printbib{Bell2011}


\subsubsection{Reviews with an individual title}
Some reviews are more detailed then others and  have their own title which should be displayed in the bibliography.
In these cases you can use the field |title| the other things stay the same.

The following entry is an example reviews the book by \citeauthor*{Welch2007}, too but with an individual title:
\begin{lstlisting}[style=bibentry,label=Hufschmid2010,caption={{@}Review\{Hufschmid2010,…\} }]
@Review{Hufschmid2010,
  author       = {Hufschmid, Thomas},
  title        = {Von Caesars \emph{theatron kynegetikon} zum \emph{amphitheatrum novum} Vespasians},
  pages        = {487--504},
  volume       = {23},
  journaltitle = JRA,		%@String used
  shortjournal = JRA-short,		%@String used
  related      = {Welch2007},
  relatedtype  = {reviewof},
  year         = {2010},
}
\end{lstlisting}
In the bibliography there will be first the individual title followed by the information of the reviewed work.

\printbib[7em]{Hufschmid2010}


\subsubsection{multiple reviewed works in one review}
Some reviews analyse several works in the same article. 
This makes no big difference for the citing or editing process.

In his review
\citeauthor{Taylor2008} not only describes the book called
 \citetitle{Welch2007} by \citeauthor{Welch2007}, 
 but at the same time compares it with \citeauthor{Sear2006}s \citetitle{Sear2006}.

The entry of the first analysed book \citetitle{Welch2007} has been described in \cref{review}.
The entry of the second reviewed book is this:
\begin{lstlisting}[style=bibentry,label=Sear2006,caption={{@}Book\{Sear2006,…\} }]
@Book{Sear2006,
  author     = {Sear, Frank},
  title      = {Roman Theatres},
  subtitle   = {An Architectural Study},
  publisher  = OUP,		%@String used
  location   = {Oxford},
  year       = {2006},
  series     = {Oxford Monographs on Classical Archaeology},
}
\end{lstlisting}

The entry of the review looks like this:
\begin{lstlisting}[style=bibentry,label=Taylor2008,caption={{@}Review\{Taylor2008,…\} }]
@Review{Taylor2008,
  author       = {Taylor, Rabun},
  number       = {3},
  pages        = {443--445},
  volume       = {67},
  journaltitle = {Journal of the Society of Architectural Historians},
  related      = {Sear2006,Welch2007},
  relatedtype  = {reviewof},
  year         = {2008},
}
\end{lstlisting}
In the field |related| you can have several \meta{bibtex-keys} which has to be separated by a comma (see line 7).

And in the bibliography all information is gathered:

\printbib[5em]{Taylor2008}


 \subsection{Type \texttt{@Thesis}}\label{thesis}
For MA and PhD thesis, which are not published a monograph or such, can be cited when they are edited as |@Thesis|.
It is important to differentiate whether it is a MA or a PhD thesis,
this can be done by |type=|\marg{|phdthesis|} or
 \marg{|mathesis|}; you also have to define the  |institution=|\marg{university}.
 
\begin{description}
\item[mandatory:] 
|author|,
|title|, |subtitle|, |titleaddon|,
|type|, |institution|,
|year|,
\item[optional:]
|doi|, |url|, |urldate|, |eprint|, |eprinttype|, |note|, |pubstate|, 
 \end{description}
 

Here is an example:
\begin{lstlisting}[style=bibentry,label=Arnolds2005,caption={{@}Thesis\{Arnolds2005,…\} }]
@Thesis{Arnolds2005,
  author      = {Markus Arnolds},
  title       = {Funktionen republikanischer und frühkaiserzeitlicher Forumsbasiliken in Italien},
  type        = {phdthesis},
  institution = {Ruprecht-Karls-Universität zu Heidelberg},
  eprint      = {urn:nbn:de:bsz:16-heidok-74406},
  eprinttype  = {urn},
  date        = {2005-05-31},
}
\end{lstlisting}

Here is the entry in the bibliography:

\printbib[5.5em]{Arnolds2005}

 
 \changes{v1.1}{2015/06/04}{Umsetzung von |@thesis| im Stil.}

 \subsection{Type \texttt{@Talk}}\label{talk}
For (oral) given papers e.\,g. at a colloquium or a proceeding we created a new entry type called ›@Talk‹.

\begin{description}
\item[mandatory:] 
|author|,
|title|, |subtitle|, |titleaddon|,
|date|,
|venue|,
|institution|,
|eventtitle|,
|eventdate|,
\item[optional:]
|doi|, |url|, |urldate|, |eprint|, |eprinttype|, |note|, |pubstate|, 
 \end{description}
 
Here is an example for a  paper  given in Berlin in 2015:
\begin{lstlisting}[style=bibentry,label=Bergmann2015,caption={{@}Talk\{Bergmann2015,…\} }]
@Talk{Bergmann2015,
  author      = {Bergmann, Birgit},
  title       = {\enquote{An exciting find}},
  date        = {2015-04-27},
  subtitle    = {Neues zum Forums-Fries der Praedia Iuliae Felicis},
  titleaddon  = {(Pompeii II, 4)},
  url         = {https://www.antikezentrum.hu-berlin.de/de/veranstaltungskalender/bibergmann},
  urldate     = {2016-05-14},
  eventtitle  = {Kolloquium der Klassischen Archäologie},
  institution = {Freie Universität Berlin},
  venue       = Berlin,
}
\end{lstlisting}
The bibliography will show the entry as:

		\printbib[6.5em]{Bergmann2015}
	
 \changes{v1.5}{2016/05/31}{Rückverweis}


\newpage
 \section{Bibliography}\label{bibliographie}
 \DescribeMacro{\printbibliography}
As long as you don’t use the option\DescribeMacro{seenote} |seenote|---for 
which a final bibliography is not needed---you will need to print you cited entries in a bibliography 
at a certain place in your document.
It can be useful to differentiate your bibliography and divide it e.\,g. into a bibliography 
with ancient authors and one with modern scholars.
Additionally you can have a bibliography with the |shorthand| shortcuts or all abbreviated journal titles, etc.

How the different bibliographies can be set up is explained now:
Let’s assume you want to have a bibliography with the ancient authors and one with modern scholars.
Since the entries of the ancient authors have the field |keyword={ancient}| (or should have it) this is done quite easy.

But first we define the heading of the whole  bibliography:
\begin{lstlisting}
\printbibheading[%
							heading=bibliography,%
							%heading=bibnumbered,% if you want it numbered
							title={Bibliography}] %heading for bibliography
\end{lstlisting}
You can give any title you would like to give (|title = |\marg{any title}).

The next step is to set up the bibliography for the ancient authors.

\begin{lstlisting}
\printbibliography[%
							keyword=ancient,%
							heading=subbibliography,
							%heading=subbibnumbered,% if you want it numbered
							title={Ancient authors and works}]
\end{lstlisting}
We tell the bibliography just to contain the entries which have have |ancient| in the field |keywords| (line 2).


Finally the bibliography for modern scholars:
\begin{lstlisting}
\printbibliography[%
							notkeyword=ancient,%
							notkeyword=corpus,%
							heading=subbibliography,
							%heading=subbibnumbered,% if you want it numbered
							title={Secondary literature}]
\end{lstlisting}
This time we exclude all entries which have |ancient| or |corpus| in the field |keywords|. 
That’s it.
(Don't be surprised about the line |notkeyword=corpus| which excludes entries with special |shorthand| labels, a further bibliography part with all the |shorthands| is described below.).


\begin{refsection}
\nocite{*}
Now have a look how it looks like with all the entries we explained above.
\begin{bsp}
\renewcommand\bibfont{\normalfont\footnotesize}
\printbibheading[%
							heading=bibliography,%
							title={Bibliography}] %heading for bibliography

\printbibliography[%
							keyword=ancient,%
							heading=subbibliography,
							title={Ancient authors and works}]

\printbibliography[%
							notkeyword=ancient,%
							notkeyword=corpus,%
							heading=subbibliography,
							title={Secondary literature}]
\end{bsp}

You can create as many bibliographies as you wish each with an other keyword if you like.
Or you can make a bibliography with all the |shorthands| used in your text---for that we use |keyword= {corpus}| (line 2):
\begin{lstlisting}
\printbibliography[%
						keyword=corpus,%
						heading=subbibliography,
					 	title={Abbreviation and corpora}]
\end{lstlisting}
Now the bibliography only lists the used entries which have |corpus| in the field |keywords|:
\begin{bsp}
\printbibliography[%
						keyword={corpus},
          				heading=subbibliography,
            			title={Abbreviation of corpora}]\label{bib:corpus}
\end{bsp}

Note: If you want to separate in your bibliography author-year labels from |shorthand| labels
 you should insure yourself that bibliography entries which contain a |shorthand| denomination 
are set with a keyword either |ancient|, |corpus| or something else, to guarantee that there is 
no bibliographical shortcut wrongly sorted in the bibliography.


Furthermore you can have a bibliography for all the abbreviated journal titles and series to have the abbreviation and its long form.
For journals it works like this:
\begin{lstlisting}
\printbiblist[%
					heading=subbibliography,
					title={Abbreviation of journals}]{shortjournal}
\end{lstlisting}

\begin{bsp}
\printbiblist[%
					heading=subbibliography,
					title={Abbreviation of journals}]{shortjournal}
\end{bsp}

For series it is done like this:
\begin{lstlisting}
\printbiblist[%
					heading=subbibliography,
					title={Abbreviation of series}]{shortseries}
\end{lstlisting}


\begin{bsp}
\printbiblist[heading=subbibliography,title={Abbreviation of series}]{shortseries}\end{bsp}
\end{refsection}




\section{Unknown work}\label{unknown}
Sometimes it is impossible to get hold of the author or editor but you still want to cite them.
If you come along such a paper you can define a |label| which will be used for citing and sorting.
This is not connected to an entry type -- it can be used with any work.
In the following example we use an entry type |@Article|:
\begin{lstlisting}[style=bibentry,label=Cosa1949,caption={{@}Article\{Cosa1949,…\} }]
@Article{Cosa1949,
  title        = {Cosa},
  subtitle     = {Republican Colony in Etruria},
  journaltitle = ClJ,
  shortjournal = ClJ-short,
  volume       = {45},
  pages        = {141--149},
  year         = {1949},
  label        = {Cosa},
  number       = {1},
}
\end{lstlisting}
 The |label| (line 9) can be defined as one wishes in this case we chose it analogous to the title: |label = {Cosa}|.
When you cite such an anonymous work it will be done like all the others:
\begin{lstlisting}
 \cite[vgl.][145--146]{Cosa1949}
\end{lstlisting}
 It will look like this:
 \begin{bsp} 
 vgl. Cosa 1949, 145--146
 \end{bsp}
and be printed in the bibliography like this:
\printbib{Cosa1949}

\subsection{Publication status}
Sometimes you know the author of an article or a book, proceedings etc. and you get a proof of the work beforehand.
You can also cite this version of the work and provide information about the publication status in the field |pubstate|.
This field is usable with all entry types provided (except |@Talk|).
There are some predefined publication status which are recommended to use since they are translated into the used language:
\begin{description}
\item[inpreparation] Typoscript is prepared for your publication. 
\begin{lstlisting}
pubstate = {inpreparation}
\end{lstlisting}
\item[submitted] Typoscript has been submitted.

\begin{lstlisting}
pubstate = {submitted}
\end{lstlisting}

\item[forthcoming] Typoscript has been accepted by the journal.

\begin{lstlisting}
pubstate = {forthcoming}
\end{lstlisting}

\item[inpress] Typoscript has been edited and you have a proof version of it.

\begin{lstlisting}
pubstate = {inpress}
\end{lstlisting}

\item[prepublished] Article has been published in an (online) preversion.

\begin{lstlisting}
pubstate = {prepublished}
\end{lstlisting}

\end{description}
This following example is about an article which was accepted by the journal and so we use |pubstate = {forthcoming}|:
\begin{lstlisting}[style=bibentry,label=Bossert:forthcoming,caption={{@}Article\{Bossert:forthcoming,…\} }]
@Article{Bossert:forthcoming,
  author       = {Lukas C. Bossert},
  title        = {\ldots\ \textsc{in formam anitqvam restitvto}?},
  subtitle     = {Überlegungen zur Inschrift der ›Porticus Deorum Consentium‹ (CIL\,VI 102) und ihren Ergänzungen im 19.{\,}Jh.},
  journaltitle = {BeStAR. Berliner Studien zum Antiken Rom},
  shortjournal = {BeStAR},
  volume       = {2},
  pubstate     = {forthcoming},
}
\end{lstlisting}

\printbib[9em]{Bossert:forthcoming}

\clearpage
\begin{multicols}{2}\footnotesize
\lstlistoflistings
\end{multicols}

\end{document}
